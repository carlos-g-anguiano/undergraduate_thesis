\chapter{Evolución temporal de un amplificador paramétrico}

Objetivos de este capítulo

\section{Óptica no lineal}


\section{Parametric Down Conversion}
El operador de compresión (\ref{eq:squeeze-op}) de un solo modo puede reescribirse en una forma análoga al operador de evolución temporal, definiendo el Hamiltoniano de una compresión
\begin{equation}\label{eq:squeeze-op-time-ev}
  \hat{S}(z) = \exp(-\frac{i}{\hbar}Ht),
\end{equation}
donde
\begin{equation} \label{eq:hamiltionian-squeeze}
  \hat{H} = i\hbar g (e^{i\beta}\hat{a}^{\dagger\,2} - e^{i\beta}\hat{a}^2),
\end{equation}
con $g=r/t$ \cite{Agarwal_2012}.
Este hamiltoniano se puede relacionar con el fenómeno físico de conversión paramétrica en descenso (PDC, por sus siglas en inglés), el cual describe la separación de un fotón de frecuencia de bombeo $\omega_p=2\omega$ en dos fotones iguales de frecuencia $\omega$ y ocurre en medios con una susceptibilidad óptica de segundo orden no lineal $\chi^{(2)}$, tal como cristales de borato de $\beta$-bario o titanilfosfato de potasio \cite{Leonhardt}. A este caso se le denomina caso degenerado de PDC, y está descrito por
\begin{equation}
  \hat{H} = \hbar (G \hat{c} \hat{a}^{\dagger\,2} + G^* \hat{c}^{\dagger}\hat{a}^2),
\end{equation}
donde el operador  en el modo de la frecuencia de bombeo está dado por $\hat{c}$y $G$ es una constante compleja que está relacionada con la susceptibilidad no lineal  del cristal. El proceso también puede separar el fotón incidente en dos fotones de distintas frecuencias $\omega_a$ y $omega_b$. Por convención a estas frecuencias se les denomina \textit{idle} y \textit{signal}, de acuerdo a la frecuencia de menor y mayor magnitud, respectivamente \cite{Loudon}. Para este caso se tiene tres operadores de distintos modos (el de bombeo, idler y signal) y se describe por
\begin{equation}
  \hat{H} = \hbar(G \hat{c} \hat{a}^{\dagger} \hat{b}^{\dagger} + G^* \hat{c}^{\dagger} \hat{a} \hat{b}),
\end{equation}
y se relacionaría de manera análoga al operador de compresión de dos modos. Este último es el caso más general o no degenerado del PDC. El proceso es de utilidad para generar estados vacíos comprimidos de uno y dos modos de manera experimental \cite{Walls}.

Para un amplificador paramétrico, el Hamiltoniano $H_{\text{PA}}$ está dado de la forma
\begin{equation}
  H_{\text{PA}} = \hbar\omega \hat{a}^{\dagger} \hat{a} - i\hbar \frac{\chi}{2}(\hat{a}^2 \exp(2i\omega t) - \hat{a}^{\dagger\,2}\exp(-2i\omega t))
\end{equation}
Las ecuaciones de movimiento de Heisenberg son
\begin{align}
  \frac{\text{d}\hat{a}}{\text{d}t} =           & \frac{1}{i\hbar}[\hat{a}, \hat{H_{\text{PA}}}] = \chi \hat{a}^\dagger   \\
  \frac{\text{d}\hat{a}^{\dagger}}{\text{d}t} = & \frac{1}{i\hbar}[\hat{a}^{\dagger}, \hat{H_{\text{PA}}}] = \chi \hat{a}
\end{align}
y la solución de estas ecuaciones es
\begin{equation}
  \hat{a}(t) = \hat{a}(0) \cosh(\chi t) + \hat{a}^\dagger(0) \senh(\chi t)
\end{equation}
Se espera entonces que la luz generada por amplificación paramétrica esté comprimida, ya que los operadores de modo son de la forma del generador de la transformación de compresión.

\section{Diagonalización de un amplificador paramétrico}
Consideremos el hamiltoniano que describe el proceso de parametric down conversion
\begin{equation*}
  \hat{H} = \omega \hat{a}^{\dagger}\hat{a}  + g \hat{a}^{2} + g^{*} \hat{a}^{\dagger\,2}
\end{equation*}
usando $\hat{a} = \hat{b} \cosh{r} + \hat{b}^{\dagger} e^{i\theta} \sinh{r}$ para que $\hat{H} = \Omega \hat{b}^{\dagger}\hat{b} + \eta$ y donde se toma $\hbar = 1$. Se muestra que $\Omega, r,\theta, \eta$ están dados por
\begin{equation*}
  \Omega = \sqrt{\omega^2 - 4|g|^2}, \quad r=\frac{1}{4}\ln{\frac{\omega - 2\sqrt{|g|^2}}{\omega + 2\sqrt{|g|^2}}}, \quad \theta = i\ln{\sqrt{\frac{g}{g^{*}}}}, \quad \eta = \frac{\Omega - \omega}{2}
\end{equation*}
Notando que $\hat{a}$ y $\hat{b}$ son operadores de bosones.

Procedemos observando que hacer el cambio de operador $\hat{a} = \hat{b} \cosh{r} + \hat{b}^{\dagger} e^{i\theta} \sinh{r}$ es equivalente a hacer la diagonalización del Hamiltoniano con el uso del operador unitario de compresión $\hat{S}(\xi)$, con $\xi = r e^{i\theta}$. De la transformación de Bogoliubov
\begin{align*}
  \hat{S}^{\dagger}(\xi) \hat{H} \hat{S}(\xi)
   & =  \hat{S}^{\dagger}(\xi) \left( \omega \hat{a}^{\dagger}\hat{a} + g \hat{a}^{2} + g^{*} \hat{a}^{\dagger\,2} \right) \hat{S}(\xi)                                                                                                                                                                                           \\
   & = \omega \hat{S}^{\dagger}(\xi) \hat{a}^{\dagger}\hat{a} \hat{S}(\xi) + g  \hat{S}^{\dagger}(\xi) \hat{a}^{2} \hat{S}(\xi) + g^{*}  \hat{S}^{\dagger}(\xi) \hat{a}^{\dagger \,2}  \hat{S}(\xi)                                                                                                                               \\
   & = \omega \hat{S}^{\dagger}(\xi) \hat{a}^{\dagger} \hat{S}(\xi)  \hat{S}^{\dagger}(\xi) \hat{a} \hat{S}(\xi)
  + g  \hat{S}^{\dagger}(\xi) \hat{a} \hat{S}(\xi) \hat{S}^{\dagger}(\xi) \hat{a} \hat{S}(\xi) + g^{*}  \hat{S}^{\dagger}(\xi) \hat{a}^{\dagger} \hat{S}(\xi) \hat{S}^{\dagger}(\xi) \hat{a}^{\dagger} \hat{S}(\xi)                                                                                                               \\
   & = \omega \hat{S}^{\dagger}(\xi) \hat{a}^{\dagger} \hat{S}(\xi)  \hat{S}^{\dagger}(\xi)\hat{a} \hat{S}(\xi) + g  \hat{S}^{\dagger}(\xi) \hat{a} \hat{S}(\xi) \hat{S}^{\dagger}(\xi) \hat{a} \hat{S}(\xi) + g^{*}  \hat{S}^{\dagger}(\xi) \hat{a}^{\dagger} \hat{S}(\xi) \hat{S}^{\dagger}(\xi) \hat{a}^{\dagger} \hat{S}(\xi) \\
   & = \omega (\hat{a}^{\dagger} \cosh{r} + \hat{a} e^{- i\theta}\sinh{r}) (\hat{a} \cosh{r} + \hat{a}^{\dagger} e^{i\theta}\sinh{r})                                                                                                                                                                                             \\
   & + g (\hat{a}\cosh{r} + \hat{a}^{\dagger} e^{i\theta}\sinh{r})^2 + g^{*} (\hat{a}^{\dagger} \cosh{r} + \hat{a} e^{-i\theta}\sinh{r})^2                                                                                                                                                                                        \\
\end{align*}

Una vez hecha la transformación del operador bosónico, se reagrupan los términos de operador común
\begin{align*}
  \hat{S}^{\dagger}(\xi) \hat{H} \hat{S}(\xi)
   & = \hat{a}^{\dagger} \hat{a} \left( \omega \cosh^2 {r} + ge^{i\theta}\cosh{r}\sinh{r} + g^{*}e^{-i\theta}\cosh{r}\sinh{r} \right) \\
   & + \hat{a} \hat{a}^{\dagger} \left( \omega \sinh^2 {r} + ge^{i\theta}\cosh{r}\sinh{r} + g^{*}e^{-i\theta}\cosh{r}\sinh{r} \right) \\
   & + \hat{a}^{\dagger\,2} (\omega e^{i\theta} \cosh{r}\sinh{r} + g e^{2i\theta}\sinh^2{r} + g^{*}\cosh^2 {r})                       \\
   & + \hat{a}^{2} (\omega e^{-i\theta} \cosh{r}\sinh{r} + g \cosh^2{r} + g^{*} e^{-2i\theta} \sinh^2 {r})
\end{align*}

Ocupando la relación de conmutación de los operadores bosónicos $\left[\hat{a}, \hat{a}^{\dagger}\right] = 1$, se obtiene que $\hat{a}\hat{a}^{\dagger} = \hat{a}^{\dagger}\hat{a}+1$ se obtiene
\begin{align*}
  \hat{S}^{\dagger}(\xi) \hat{H} \hat{S}(\xi)
   & = \hat{a}^{\dagger} \hat{a} \left( \omega (\cosh^2 {r} + \sinh^2 {r}) + 2ge^{i\theta}\cosh{r}\sinh{r} + 2g^{*}e^{-i\theta}\cosh{r}\sinh{r} \right) \\
   & + \left( \omega \sinh^2 {r} + ge^{i\theta}\cosh{r}\sinh{r} + g^{*}e^{-i\theta}\cosh{r}\sinh{r} \right)                                             \\
   & + \hat{a}^{\dagger\,2} (\omega e^{i\theta} \cosh{r}\sinh{r} + g e^{2i\theta}\sinh^2{r} + g^{*}\cosh^2 {r})                                         \\
   & + \hat{a}^{2} (\omega e^{-i\theta} \cosh{r}\sinh{r} + g^{*} e^{-2i\theta} \sinh^2 {r} + g \cosh^2{r})
\end{align*}

para llegar a la forma del Hamiltoniano del oscilador armónico $\hat{H} = \Omega \hat{a}^{\dagger}\hat{a} + \eta$ se ha de cumplir que
\begin{align*}
  \omega e^{i\theta} \cosh{r}\sinh{r} + g e^{2i\theta}\sinh^2{r} + g^{*}\cosh^{2} {r}   & = 0 \\
  \omega e^{-i\theta} \cosh{r}\sinh{r} + g^{*} e^{-2i\theta} \sinh^2 {r} + g \cosh^2{r} & = 0
\end{align*}
dividiendo entre $\cosh^2{r}$ ambas ecuaciones, obtenemos un par de ecuaciones cuadráticas para $\tanh{r}$
\begin{align*}
  \omega e^{i\theta} \tanh{r} + g e^{2i\theta} \tanh^2 {r} + g^{*}   & = 0 \\
  \omega e^{-i\theta} \tanh{r} + g^{*} e^{-2i\theta} \tanh^2 {r} + g & = 0
\end{align*}
Resolviendo la ecuación cuadrática para la tangente hiperbólica
\begin{align*}
  \tanh{r} & = \frac{-\omega\pm \sqrt{\omega^2 - 4|g|^2}}{2ge^{i\theta}}      \\
  \tanh{r} & = \frac{-\omega\pm \sqrt{\omega^2 - 4|g|^2}}{2g^{*}e^{-i\theta}}
\end{align*}
Que la igualdad entre valores de $\tanh{r}$, se debe cumplir que el denominador sea el mismo en ambos casos
\begin{equation*}
  2 g e^{i\theta} = 2 g^{*} e^{-i\theta}
\end{equation*}
de donde se deduce el valor de theta
\begin{align*}
  e^{-2i\theta} & = \frac{g}{g^{*}}                         \\
  e^{-i\theta}  & = \sqrt{\frac{g}{g^{*}}}                  \\
  -i \theta     & = \ln\left(\sqrt{\frac{g}{g^{*}}} \right)
\end{align*}
y finalmente
\begin{equation*}
  \theta = i\ln \left(\sqrt{\frac{g}{g^{*}}} \right)
\end{equation*}
Por calcular $r$, se tiene, usando $e^{i\theta} = \sqrt{\frac{g^{*}}{g}}$,
\begin{equation*}
  \tanh{r} = \frac{-\omega \pm \sqrt{\omega^2 - 4|g|^2}}{2g\left( \sqrt{\frac{g^*}{g}} \right)} = \frac{-\omega \pm \sqrt{\omega^2 - 4|g|^2}}{2 \vert g \vert}
\end{equation*}
como $r\geq 0$ tenemos que $\tanh{r} \geq 0$ luego
\begin{equation*}
  \tanh{r} = \frac{-\omega + \sqrt{\omega^2 - 4|g|^2}}{2 \vert g \vert}
\end{equation*}
Obteniendo $\tanh^{-1}$ usando
\begin{equation*}
  \tanh^{-1}{x} = \frac{1}{2}\ln{\left( \frac{1+x}{1-x} \right)}
\end{equation*}
despejamos $r$
\begin{align*}
  r & =\frac{1}{2} \ln \left\{ \frac{\displaystyle{1 - \frac{\omega}{2\vert g\vert} + \frac{\sqrt{\omega^{2}-4\vert g\vert^{2}}}{2\vert g\vert}}}{\displaystyle{1+ \frac{\omega}{2\vert g\vert} - \frac{\sqrt{\omega^{2}-4\vert g\vert^{2}}}{2\vert g\vert}}} \right\}                         \\
    & =  \frac{1}{2} \ln \left\{ \frac{2\vert g \vert -\omega + \sqrt{\omega^{2}-4\vert g\vert^{2}}}{2\vert g \vert + \omega - \sqrt{\omega^{2}-4\vert g\vert^{2}}} \right\}                                                                                                                   \\
    & =  \frac{1}{2} \ln \left\{ \frac{-\sqrt{\omega - 2\vert g\vert}\sqrt{\omega - 2\vert g\vert} + \sqrt{\omega-2\vert g\vert}\sqrt{\omega+2\vert g\vert}}{\sqrt{\omega + 2\vert g\vert}\sqrt{\omega + 2\vert g\vert} - \sqrt{\omega-2\vert g\vert}\sqrt{\omega^{2}+2\vert g\vert}} \right\} \\
    & =  \frac{1}{2} \ln \left\{ \frac{\sqrt{\omega - 2\vert g\vert}\left(\sqrt{\omega + 2\vert g\vert} - \sqrt{\omega-2\vert g\vert}\right)}{\sqrt{\omega + 2\vert g\vert}\left(\sqrt{\omega + 2\vert g\vert} - \sqrt{\omega-2\vert g\vert}\right)} \right\}                                  \\
    & =  \frac{1}{4} \ln \left\{ \frac{\omega - 2\vert g\vert}{\omega + 2\vert g\vert} \right\}                                                                                                                                                                                                \\
\end{align*}
Con esta condición, se obtiene el Hamiltoniano en forma de oscilador armónico

\begin{align*}
  \hat{S}^{\dagger}(\xi) \hat{H} \hat{S}(\xi) & = \hat{a}^{\dagger}\hat{a} \left( \omega (\cosh^2 {r} + \sinh^2 {r}) + 2\cosh{r}\sinh{r} \left[ g e^{i\theta} + g^{*} e^{-i\theta} \right] \right) \\
                                              & + \left( \omega \sinh^2 {r} + g e^{i\theta}\cosh{r}\sinh{r} + g^{*}e^{-i\theta}\cosh{r}\sinh{r} \right)                                            \\
                                              & = \hat{a}^{\dagger}\hat{a} \left( \omega \cosh{2r} + \sinh{2r} \left[ g e^{i\theta} + g^{*} e^{-i\theta} \right] \right)                           \\
                                              & + \left( \omega \sinh^2 {r} + g e^{i\theta}\cosh{r}\sinh{r} + g^{*}e^{-i\theta}\cosh{r}\sinh{r} \right)                                            \\
\end{align*}


usando la identidad $\sinh^{2}r = (\cosh{2r}-1)/2$ y empleandtenemos que
\begin{align*}
  \hat{S}^{\dagger}(\xi) \hat{H} \hat{S}(\xi) & = \hat{a}^{\dagger}\hat{a} \left( \omega \cosh{2r} + \sinh{2r} \left[ g \sqrt{\frac{g^{*}}{g}} + g^{*} \sqrt{\frac{g}{g^{*}}} \right] \right)                              \\
                                              & + \frac{\omega}{2} \cosh{2r} + g \sqrt{\frac{g^{*}}{g}} \cosh{r}\sinh{r} + g^{*}\sqrt{\frac{g}{g^{*}}}\cosh{r}\sinh{r}-\frac{\omega}{2}                                    \\
                                              & = \hat{a}^{\dagger}\hat{a} \left( \omega \cosh{2r} + 2\vert g\vert \sinh{2r} \right) + \frac{1}{2}\left(\omega\cosh{2r} + 2\vert g \vert \sinh{2r}\right)-\frac{\omega}{2} \\
\end{align*}

Empleando el hecho de
\begin{equation*}
  e^{2r} = \sqrt{\frac{\omega-2\vert g\vert}{\omega+2\vert g\vert}}
\end{equation*}
\begin{align*}
  \cosh{2r} & = \frac{e^{2r}}{2} + \frac{e^{-2r}}{2} = \frac{1}{2} \sqrt{\frac{\omega-2\vert g\vert}{\omega+2\vert g\vert}} + \frac{1}{2} \sqrt{\frac{\omega+2\vert g\vert}{\omega-2\vert g\vert}} \\
            & = \frac{1}{2} \frac{\omega-2\vert g\vert + \omega + 2\vert g\vert}{\sqrt{\omega^{2}-4\vert g\vert^{2}}} = \frac{\omega}{\sqrt{\omega^{2}-4\vert g\vert^{2}}}
\end{align*}
\begin{align*}
  \sinh{2r} & = \frac{e^{2r}}{2} - \frac{e^{-2r}}{2} = \frac{1}{2} \sqrt{\frac{\omega-2\vert g\vert}{\omega+2\vert g\vert}} - \frac{1}{2} \sqrt{\frac{\omega+2\vert g\vert}{\omega-2\vert g\vert}} \\
            & = \frac{1}{2} \frac{\omega-2\vert g\vert - \omega - 2\vert g\vert}{\sqrt{\omega^{2}-4\vert g\vert^{2}}} = - \frac{2\vert g\vert}{\sqrt{\omega^{2}-4\vert g\vert^{2}}}
\end{align*}
entonces
\begin{align*}
  \omega \cosh{2r} + 2\vert g\vert \sinh{2r} = \frac{\omega^{2}}{\sqrt{\omega^{2}-4\vert g\vert^{2}}} - \frac{4\vert g\vert^{2}}{\sqrt{\omega^{2}-4\vert g\vert^{2}}} = \sqrt{\omega^{2}-4\vert g\vert^{2}}
\end{align*}
ahora si definimos
\begin{equation*}
  \Omega = \sqrt{\omega^{2}-4\vert g\vert^{2}}
\end{equation*}
\begin{equation*}
  \eta = \frac{\Omega - \omega}{2}
\end{equation*}
Finalmente, la transformación de similitud se convierte en
\begin{equation*}
  \hat{S}^{\dagger}(\xi) \hat{H} \hat{S}(\xi) = \Omega \hat{a}^{\dagger}\hat{a} + \eta
\end{equation*}
\subsection{Evolución del estado inicial}
Consideremos un estado inicial del campo
\begin{equation*}
  \vert \psi(0)\rangle = A \left(\vert \alpha \rangle + c \vert  \beta\rangle \right)
\end{equation*}
con
\begin{equation*}
  A = \frac{1}{\sqrt{1+\vert c\vert^{2} + 2 \mathrm{Re}\left(c\langle \alpha \vert \beta \rangle \right)}}
\end{equation*}
entonces
\begin{align*}
  \vert \psi(t) \rangle & = \hat{S}(\xi)\hat{S}^{\dagger}(\xi) e^{-it \hat{H}} \hat{S}(\xi)\hat{S}^{\dagger}(\xi)\vert \psi(0)\rangle = \hat{S}(\xi)\exp\left(-it \hat{S}^{\dagger}(\xi)\hat{H}\hat{S}(\xi)\right) \hat{S}^{\dagger}(\xi)\vert \psi(0)\rangle                                                                                                                                                     \\
                        & = \hat{S}(\xi)\exp\left(-it \left(\Omega \,\hat{a}^{\dagger}\hat{a} + \eta\right)\right) \hat{S}^{\dagger}(\xi)\vert \psi(0)\rangle = e^{-i\eta t} \hat{S}(\xi)e^{-it \Omega \,\hat{a}^{\dagger}\hat{a}} \hat{S}^{\dagger}(\xi)\vert \psi(0)\rangle                                                                                                                                     \\
                        & = e^{-i\eta t}e^{-it \Omega \,\hat{a}^{\dagger}\hat{a}} e^{it \Omega \,\hat{a}^{\dagger}\hat{a}} \hat{S}(\xi) e^{-it \Omega \,\hat{a}^{\dagger}\hat{a}} \hat{S}^{\dagger}(\xi)\vert \psi(0)\rangle                                                                                                                                                                                      \\
                        & = e^{-i\eta t}e^{-it \Omega \,\hat{a}^{\dagger}\hat{a}} e^{it \Omega \,\hat{a}^{\dagger}\hat{a}} \exp\left(\xi \frac{\hat{a}^{\dagger\,2}}{2} - \xi^{*} \frac{\hat{a}^{2}}{2}\right) e^{-it \Omega \,\hat{a}^{\dagger}\hat{a}} \hat{S}^{\dagger}(\xi)\vert \psi(0)\rangle                                                                                                               \\
                        & = e^{-i\eta t} e^{-it \Omega \,\hat{a}^{\dagger}\hat{a}}  \exp\left(\xi e^{it \Omega \,\hat{a}^{\dagger}\hat{a}}\frac{\hat{a}^{\dagger\,2}}{2}e^{-it \Omega \,\hat{a}^{\dagger}\hat{a}} - \xi^{*} e^{it \Omega \,\hat{a}^{\dagger}\hat{a}} \frac{\hat{a}^{2}}{2} e^{-it \Omega \,\hat{a}^{\dagger}\hat{a}}\right)  \hat{S}^{\dagger}(\xi)\vert \psi(0)\rangle                           \\
                        & = e^{-i\eta t} e^{-it \Omega \,\hat{a}^{\dagger}\hat{a}}  \exp\left(\frac{\xi}{2}\left(e^{it \Omega \,\hat{a}^{\dagger}\hat{a}}\hat{a}^{\dagger}e^{-it \Omega \,\hat{a}^{\dagger}\hat{a}}\right)^{2} - \frac{\xi^{*}}{2}\left(e^{it \Omega \,\hat{a}^{\dagger}\hat{a}} \hat{a} e^{-it \Omega \,\hat{a}^{\dagger}\hat{a}}\right)^{2} \right)  \hat{S}^{\dagger}(\xi)\vert \psi(0)\rangle \\
\end{align*}
pero
\begin{align*}
  e^{i\Omega t \,\hat{a}^{\dagger}\hat{a}}\hat{a} e^{-i\Omega t \,\hat{a}^{\dagger}\hat{a}} & = \hat{a} + \left[i\Omega t\hat{a}^{\dagger}\hat{a},\hat{a}\right] + \frac{1}{2!} \left[i\Omega t\hat{a}^{\dagger}\hat{a},\left[i\Omega t\hat{a}^{\dagger}\hat{a},\hat{a}\right]\right]+\cdots \\
                                                                                            & = \left(1-i\Omega\,t+\frac{(-i\Omega\,t)^{2}}{2!}+\cdots\right)\hat{a} = e^{-i\Omega t} \hat{a}
\end{align*}
y además
\begin{align*}
  e^{i\Omega t \,\hat{a}^{\dagger}\hat{a}}\hat{a}^{\dagger} e^{-i\Omega t \,\hat{a}^{\dagger}\hat{a}} & = \left(e^{i\Omega t \,\hat{a}^{\dagger}\hat{a}}\hat{a} e^{-i\Omega t \,\hat{a}^{\dagger}\hat{a}}\right)^{\dagger} = \left(e^{-i\Omega t}\hat{a}\right)^{\dagger} = e^{i\Omega t}\hat{a}^{\dagger}
\end{align*}
luego
\begin{align*}
  \vert \psi(t) \rangle & = e^{-i\eta t} e^{-it \Omega \,\hat{a}^{\dagger}\hat{a}}  \exp\left(\frac{\xi}{2}\left(e^{it \Omega \,\hat{a}^{\dagger}\hat{a}}\hat{a}^{\dagger}e^{-it \Omega \,\hat{a}^{\dagger}\hat{a}}\right)^{2} - \frac{\xi^{*}}{2}\left(e^{it \Omega \,\hat{a}^{\dagger}\hat{a}} \hat{a} e^{-it \Omega \,\hat{a}^{\dagger}\hat{a}}\right)^{2} \right)  \hat{S}^{\dagger}(\xi)\vert \psi(0)\rangle \\
                        & = e^{-i\eta t} e^{-it \Omega \,\hat{a}^{\dagger}\hat{a}}  \exp\left(\xi e^{i2\Omega t} \frac{\hat{a}^{\dagger\,2}}{2} - \xi^{*} e^{-i2\Omega t} \frac{\hat{a}^{2}}{2}  \right)  \hat{S}^{\dagger}(\xi)\vert \psi(0)\rangle = e^{-i\eta t} e^{-it \Omega \,\hat{a}^{\dagger}\hat{a}} \hat{S}\left(\xi e^{i2\Omega t} \right)  \hat{S}^{\dagger}(\xi)\vert \psi(0)\rangle
\end{align*}
\subsection{Funci\'on de Wigner del campo}
De la expresión (\ref{eq:c4-moya}) se puede escribir la función de Wigner como
\begin{align*}
  W(\gamma,t) & = \mathrm{Tr}\left\{\hat{\rho}\frac{2}{\pi}\hat{D}(\gamma)(-1)^{\hat{a}^{\dagger}\hat{a}}\hat{D}^{\dagger}(\gamma)\right\} = \mathrm{Tr}\left\{\vert\psi(t) \rangle\langle \psi(t)\vert\frac{2}{\pi}\hat{D}(\gamma)(-1)^{\hat{a}^{\dagger}\hat{a}}\hat{D}^{\dagger}(\gamma)\right\}
  \\
              & = \frac{2}{\pi} \langle \psi(t)\vert \hat{D}(\gamma)(-1)^{\hat{a}^{\dagger}\hat{a}}\hat{D}^{\dagger}(\gamma) \vert \psi(t)\rangle                                                                                                                                                                                                                                                                                      \\
              & = \frac{2}{\pi} \langle \psi(0)\vert \hat{S}(\xi)\hat{S}^{\dagger}\left(\xi e^{i2\Omega t}\right)e^{i\Omega t \hat{a}^{\dagger}\hat{a}} e^{i\eta t} \hat{D}(\gamma)(-1)^{\hat{a}^{\dagger}\hat{a}}\hat{D}^{\dagger}(\gamma)e^{-i\eta t} e^{-i\Omega t\hat{a}^{\dagger}\hat{a}}\hat{S}(\xi e^{i 2\Omega t})\hat{S}^{\dagger}(\xi) \vert \psi(0)\rangle
  \\
              & = \frac{2}{\pi} \langle \psi(0)\vert \hat{S}(\xi)\hat{S}^{\dagger}\left(\xi e^{i2\Omega t}\right)e^{i\Omega t \hat{a}^{\dagger}\hat{a}} \hat{D}(\gamma)e^{-i\Omega t\hat{a}^{\dagger}\hat{a}}(-1)^{\hat{a}^{\dagger}\hat{a}}e^{i\Omega t\hat{a}^{\dagger}\hat{a}}\hat{D}^{\dagger}(\gamma) e^{-i\Omega t\hat{a}^{\dagger}\hat{a}}\hat{S}(\xi e^{i 2\Omega t})\hat{S}^{\dagger}(\xi) \vert \psi(0)\rangle
  \\
              & = \frac{2}{\pi} \langle \psi(0)\vert \hat{S}(\xi)\hat{S}^{\dagger}\left(\xi e^{i2\Omega t}\right)e^{i\Omega t \hat{a}^{\dagger}\hat{a}} \hat{D}(\gamma)e^{-i\Omega t\hat{a}^{\dagger}\hat{a}}(-1)^{\hat{a}^{\dagger}\hat{a}} \left(e^{i\Omega t\hat{a}^{\dagger}\hat{a}}\hat{D}(\gamma) e^{-i\Omega t\hat{a}^{\dagger}\hat{a}}\right)^{\dagger}\hat{S}(\xi e^{i 2\Omega t})\hat{S}^{\dagger}(\xi) \vert \psi(0)\rangle
  \\
\end{align*}
por otra parte
\begin{align*}
  e^{i\,\Omega t \,\hat{a}^{\dagger}\hat{a}} \hat{D}(\gamma) e^{-i\,\Omega t \,\hat{a}^{\dagger}\hat{a}} & = e^{i\,\Omega t \,\hat{a}^{\dagger}\hat{a}} \exp\left(\gamma\hat{a}^{\dagger}-\gamma^{*}\hat{a}\right) e^{-i\,\Omega t \,\hat{a}^{\dagger}\hat{a}}                                                                                                        \\
                                                                                                         & = \exp\left(\gamma \left(e^{i\,\Omega t \,\hat{a}^{\dagger}\hat{a}} \hat{a} e^{-i\,\Omega t \,\hat{a}^{\dagger}\hat{a}}\right)^{\dagger} -\gamma^{*} e^{i\,\Omega t \,\hat{a}^{\dagger}\hat{a}} \hat{a} e^{-i\,\Omega t \,\hat{a}^{\dagger}\hat{a}}\right) \\
                                                                                                         & = \exp\left(\gamma \left(\hat{a}e^{-i\Omega t}\right)^{\dagger} -\gamma^{*} \hat{a}e^{-i\Omega t}\right) = \hat{D}\left(\gamma e^{i\Omega t}\right)                                                                                                        \\
\end{align*}
por lo que la funci\'on de Wigner se escribe como
\begin{align*}
  W(\gamma,t) & = \frac{2}{\pi} \langle \psi(0)\vert \hat{S}(\xi)\hat{S}^{\dagger}\left(\xi e^{i2\Omega t}\right) \hat{D}\left(\gamma e^{i\Omega t}\right) (-1)^{\hat{a}^{\dagger}\hat{a}} \hat{D}^{\dagger}\left(\gamma e^{i\Omega t}\right) \hat{S}(\xi e^{i 2\Omega t})\hat{S}^{\dagger}(\xi) \vert \psi(0)\rangle
  \\
              & = \frac{2}{\pi} \langle \psi(0)\vert \hat{S}(\xi)\hat{S}^{\dagger}\left(\xi e^{i2\Omega t}\right) \hat{D}\left(\gamma e^{i\Omega t}\right) (-1)^{\hat{a}^{\dagger}\hat{a}}\hat{S}(\xi e^{i 2\Omega t}) \hat{S}^{\dagger}(\xi e^{i 2\Omega t})\hat{D}^{\dagger}\left(\gamma e^{i\Omega t}\right) \hat{S}(\xi e^{i 2\Omega t})\hat{S}^{\dagger}(\xi) \vert \psi(0)\rangle
  \\
\end{align*}
pero
\begin{align*}
  (-1)^{\hat{a}^{\dagger}\hat{a}} \hat{S}\left(\xi e^{i2\Omega t}\right) & =
  e^{i\pi \hat{a}^{\dagger}\hat{a}} \hat{S}\left(\xi e^{i2\Omega t}\right) e^{-i\pi \hat{a}^{\dagger}\hat{a}} e^{i\pi\hat{a}^{\dagger}\hat{a}} = e^{i\pi \hat{a}^{\dagger}\hat{a}} \exp\left(\xi e^{i2\Omega t}\frac{\hat{a}^{\dagger\,2}}{2} - \xi^{*} e^{- i2\Omega t}\frac{\hat{a}^{2}}{2}\right) e^{-i\pi \hat{a}^{\dagger}\hat{a}} e^{i\pi\hat{a}^{\dagger}\hat{a}}
  \\
                                                                         & = \exp\left(\xi e^{i2\Omega t} e^{i\pi \hat{a}^{\dagger}\hat{a}}\frac{\hat{a}^{\dagger\,2}}{2} e^{-i\pi \hat{a}^{\dagger}\hat{a}} - \xi^{*} e^{- i2\Omega t} e^{i\pi \hat{a}^{\dagger}\hat{a}} \frac{\hat{a}^{2}}{2} e^{-i\pi \hat{a}^{\dagger}\hat{a}} \right) e^{i\pi\hat{a}^{\dagger}\hat{a}} \\
                                                                         & = \exp\left(\xi e^{i2\Omega t} e^{i 2\pi}\frac{\hat{a}^{\dagger\,2}}{2} - \xi^{*} e^{- i2\Omega t}e^{-i2\pi} \frac{\hat{a}^{2}}{2} \right) e^{i\pi\hat{a}^{\dagger}\hat{a}}                                                                                                                      \\
                                                                         & = \hat{S}\left(\xi e^{i2\Omega t}e^{i2\pi}\right) e^{i\pi\hat{a}^{\dagger}\hat{a}} = \hat{S}\left(\xi e^{i2\Omega t}\right) e^{i\pi\hat{a}^{\dagger}\hat{a}}
\end{align*}
entonces
\begin{align*}
  W(\gamma,t) & = \frac{2}{\pi} \langle \psi(0)\vert \hat{S}(\xi)\hat{S}^{\dagger}\left(\xi e^{i2\Omega t}\right) \hat{D}\left(\gamma e^{i\Omega t}\right) \hat{S}(\xi e^{i 2\Omega t}) (-1)^{\hat{a}^{\dagger}\hat{a}}\left(\hat{S}^{\dagger}(\xi e^{i 2\Omega t})\hat{D}\left(\gamma e^{i\Omega t}\right) \hat{S}(\xi e^{i 2\Omega t})\right)^{\dagger}\hat{S}^{\dagger}(\xi) \vert \psi(0)\rangle
  \\
\end{align*}
por otra parte
\begin{align*}
  \hat{S}^{\dagger}\left(\xi e^{i2\Omega t}\right)\hat{D}\left(\gamma e^{i\Omega t}\right) \hat{S}(\xi e^{i 2\Omega t}) & = \hat{S}^{\dagger}\left(\xi'\right)\hat{D}\left(\gamma' \right) \hat{S}(\xi') =
  \hat{S}^{\dagger}(\xi') \exp\left(\gamma'\hat{a}^{\dagger}-\gamma'^{*}\hat{a}\right)\hat{S}(\xi')
  \\
                                                                                                                        & = \exp\left(\gamma' \hat{S}^{\dagger}(\xi')\hat{a}^{\dagger}\hat{S}(\xi')-\gamma'^{*}\hat{S}^{\dagger}(\xi')\hat{a}\hat{S}(\xi')\right)               \\
                                                                                                                        & =  \exp\left(\gamma' \left(\hat{S}^{\dagger}(\xi')\hat{a}\hat{S}(\xi')\right)^{\dagger}-\gamma'^{*}\hat{S}^{\dagger}(\xi')\hat{a}\hat{S}(\xi')\right)
\end{align*}
pero
\begin{align*}
  \hat{S}^{\dagger}(\xi')\hat{a}\hat{S}(\xi') = \mu' \hat{a} + \nu' \hat{a}^{\dagger}
\end{align*}
entonces
\begin{align*}
  \hat{S}^{\dagger}\left(\xi e^{i2\Omega t}\right)\hat{D}\left(\gamma e^{i\Omega t}\right) \hat{S}(\xi e^{i 2\Omega t})
   & = \exp\left(\gamma' \left(\mu' \hat{a}+\nu'\hat{a}^{\dagger}\right)^{\dagger}-\gamma'^{*}\left(\mu' \hat{a}+\nu'\hat{a}^{\dagger}\right)\right) \\
   & = \exp\left(\gamma' \left(\mu' \hat{a}^{\dagger}+\nu'^{*}\hat{a}\right)-\gamma'^{*}\left(\mu' \hat{a}+\nu'\hat{a}^{\dagger}\right)\right)       \\
   & = \exp\left(\left(\mu'\gamma'-\nu'\gamma'^{*} \right)\hat{a}^{\dagger}-\left(\mu'\gamma'-\nu'\gamma'^{*} \right)^{*}\hat{a}\right)
  = \hat{D}\left(\mu'\gamma'-\nu'\gamma'^{*}\right)
\end{align*}
por otra parte
\begin{align*}
  \mu' = \cosh\vert\xi'\vert = \cosh\left\vert \xi e^{i 2\Omega t}\right\vert = \cosh\left\vert \xi\right\vert = \mu
\end{align*}
\begin{align*}
  \nu' = \frac{\xi'}{\vert\xi'\vert}\sinh\left\vert \xi'\right\vert = \frac{\xi e^{i2\Omega t}}{\left\vert\xi e^{i2\Omega t}\right\vert} \sinh\left\vert \xi e^{i2\Omega t} \right\vert = e^{i2\Omega t} \frac{\xi}{\left\vert\xi \right\vert} \sinh\left\vert \xi  \right\vert = \nu \, e^{i2\Omega t}
\end{align*}
luego
\begin{align*}
  \hat{S}^{\dagger}\left(\xi e^{i2\Omega t}\right)\hat{D}\left(\gamma e^{i\Omega t}\right) \hat{S}(\xi e^{i 2\Omega t})
   & = \hat{D}\left(\mu \gamma e^{i\Omega t} - \nu e^{i2\Omega t} \gamma^{*}e^{-i\Omega t}\right) = \hat{D}\left(\left(\mu\gamma - \nu\gamma^{*}\right)e^{i\Omega t}\right)
\end{align*}
por lo que la funci\'on de Wigner se escribe como
\begin{align*}
  W(\gamma,t) & = \frac{2}{\pi} \langle \psi(0)\vert \hat{S}(\xi)\hat{D}\left(\left(\mu\gamma-\nu\gamma^{*}\right)e^{i\Omega t}\right) (-1)^{\hat{a}^{\dagger}\hat{a}} \hat{D}^{\dagger}\left(\left(\mu\gamma-\nu\gamma^{*}\right)e^{i\Omega t}\right)\hat{S}^{\dagger}(\xi) \vert \psi(0)\rangle
  \\
              & = \frac{2}{\pi} \langle \psi(0)\vert \hat{S}(\xi)\hat{D}\left(\left(\mu\gamma-\nu\gamma^{*}\right)e^{i\Omega t}\right) \hat{S}^{\dagger}(\xi)\hat{S}(\xi)(-1)^{\hat{a}^{\dagger}\hat{a}} \hat{D}^{\dagger}\left(\left(\mu\gamma-\nu\gamma^{*}\right)e^{i\Omega t}\right)\hat{S}^{\dagger}(\xi) \vert \psi(0)\rangle
\end{align*}
pero
\begin{align*}
  \hat{S}(\xi)(-1)^{\hat{a}^{\dagger}\hat{a}} & = \hat{S}(\xi)e^{i\pi\hat{a}^{\dagger}\hat{a}} = e^{i\pi\hat{a}^{\dagger}\hat{a}}e^{-i\pi\hat{a}^{\dagger}\hat{a}} \hat{S}(\xi) e^{i\pi\hat{a}^{\dagger}\hat{a}} = e^{i\pi\hat{a}^{\dagger}\hat{a}}e^{-i\pi\hat{a}^{\dagger}\hat{a}} \exp\left(\frac{\xi}{2}\hat{a}^{\dagger\,2} - \frac{\xi^{*}}{2}\hat{a}^{2} \right) e^{i\pi\hat{a}^{\dagger}\hat{a}} \\
                                              & = e^{i\pi\hat{a}^{\dagger}\hat{a}} \exp\left(\frac{\xi}{2}e^{-i\pi\hat{a}^{\dagger}\hat{a}}\hat{a}^{\dagger\,2}e^{i\pi\hat{a}^{\dagger}\hat{a}} - \frac{\xi^{*}}{2}e^{-i\pi\hat{a}^{\dagger}\hat{a}}\hat{a}^{2}e^{i\pi\hat{a}^{\dagger}\hat{a}} \right)
  = e^{i\pi\hat{a}^{\dagger}\hat{a}} \exp\left(\frac{\xi}{2} \hat{a}^{\dagger\,2} e^{-i2\pi} - \frac{\xi^{*}}{2} \hat{a}^{2}e^{i2\pi} \right)                                                                                                                                                                                                                                                            \\
                                              & = e^{i\pi\hat{a}^{\dagger}\hat{a}} \exp\left(\frac{\xi}{2} \hat{a}^{\dagger\,2}  - \frac{\xi^{*}}{2} \hat{a}^{2} \right) =  e^{i\pi\hat{a}^{\dagger}\hat{a}} \hat{S}(\xi) = (-1)^{\hat{a}^{\dagger}\hat{a}} \hat{S}(\xi)
\end{align*}
leugo la funci\'on de Wigner
\begin{align*}
  W(\gamma,t)
   & = \frac{2}{\pi} \langle \psi(0)\vert \hat{S}(\xi)\hat{D}\left(\left(\mu\gamma-\nu\gamma^{*}\right)e^{i\Omega t}\right) \hat{S}^{\dagger}(\xi)(-1)^{\hat{a}^{\dagger}\hat{a}} \hat{S}(\xi)\hat{D}^{\dagger}\left(\left(\mu\gamma-\nu\gamma^{*}\right)e^{i\Omega t}\right)\hat{S}^{\dagger}(\xi) \vert \psi(0)\rangle              \\
   & = \frac{2}{\pi} \langle \psi(0)\vert \hat{S}(\xi)\hat{D}\left(\left(\mu\gamma-\nu\gamma^{*}\right)e^{i\Omega t}\right) \hat{S}^{\dagger}(\xi)(-1)^{\hat{a}^{\dagger}\hat{a}} \left(\hat{S}(\xi)\hat{D}\left(\left(\mu\gamma-\nu\gamma^{*}\right)e^{i\Omega t}\right)\hat{S}^{\dagger}(\xi)\right)^{\dagger} \vert \psi(0)\rangle
\end{align*}
pero
\begin{align*}
  \hat{S}(\xi) \hat{D}\left(\left(\mu\gamma-\nu\gamma^{*}\right)e^{i\Omega t}\right)\hat{S}^{\dagger}(\xi) & = \hat{S}(\xi) \exp\left(\left(\mu\gamma-\nu\gamma^{*}\right)e^{i\Omega t}\hat{a}^{\dagger}-\left(\mu\gamma^{*}-\nu^{*}\gamma\right)e^{-i\Omega t}\hat{a}\right)\hat{S}^{\dagger}(\xi)                                                            \\
                                                                                                           & = \exp\left(\left(\mu\gamma-\nu\gamma^{*}\right)e^{i\Omega t}\left(\hat{S}(\xi)\hat{a}\hat{S}^{\dagger}(\xi)\right)^{\dagger}-\left(\mu\gamma^{*}-\nu^{*}\gamma\right)e^{-i\Omega t}\left(\hat{S}(\xi)\hat{a}\hat{S}^{\dagger}(\xi)\right)\right) \\
\end{align*}
pero
\begin{align*}
  \hat{S}(\xi)\hat{a}\hat{S}^{\dagger}(\xi) = \mu \hat{a} - \nu \hat{a}^{\dagger}
\end{align*}
por lo que
\begin{align*}
  \hat{S}(\xi) \hat{D}\left(\left(\mu\gamma-\nu\gamma^{*}\right)e^{i\Omega t}\right)\hat{S}^{\dagger}(\xi)
   & = \exp\left(\left(\mu\gamma-\nu\gamma^{*}\right)e^{i\Omega t}\left(\hat{S}(\xi)\hat{a}\hat{S}^{\dagger}(\xi)\right)^{\dagger}-\left(\mu\gamma^{*}-\nu^{*}\gamma\right)e^{-i\Omega t}\left(\hat{S}(\xi)\hat{a}\hat{S}^{\dagger}(\xi)\right)\right) \\
   & = \exp\left(\left(\mu\gamma-\nu\gamma^{*}\right)e^{i\Omega t}\left(\mu \hat{a} - \nu \hat{a}^{\dagger}\right)^{\dagger}-\left(\mu\gamma^{*}-\nu^{*}\gamma\right)e^{-i\Omega t}\left(\mu \hat{a} - \nu \hat{a}^{\dagger}\right)\right)             \\
   & = \exp\left(\left(\mu\gamma-\nu\gamma^{*}\right)e^{i\Omega t}\left(\mu \hat{a}^{\dagger} - \nu^{*} \hat{a}\right)-\left(\mu\gamma^{*}-\nu^{*}\gamma\right)e^{-i\Omega t}\left(\mu \hat{a} - \nu \hat{a}^{\dagger}\right)\right)                   \\
   & = \exp\left(\left(\mu\left(\mu\gamma-\nu\gamma^{*}\right)e^{i\Omega t}+\nu\left(\mu\gamma^{*}-\nu^{*}\gamma\right)e^{-i\Omega t}\right)\hat{a}^{\dagger} \right.                                                                                  \\
   & \left.- \left(\mu\left(\mu\gamma-\nu\gamma^{*}\right)e^{i\Omega t}+\nu\left(\mu\gamma^{*}-\nu^{*}\gamma\right)e^{-i\Omega t}\right)^{*}\hat{a}\right)                                                                                             \\
   & = \hat{D}\left(\mu\left(\mu\gamma-\nu\gamma^{*}\right)e^{i\Omega t}+\nu\left(\mu\gamma^{*}-\nu^{*}\gamma\right)e^{-i\Omega t}\right)
\end{align*}
finalmente
\begin{align*}
  W(\gamma,t)
   & = \frac{2}{\pi} \langle \psi(0)\vert \hat{D}\left(\mu\left(\mu\gamma-\nu\gamma^{*}\right)e^{i\Omega t}+\nu\left(\mu\gamma^{*}-\nu^{*}\gamma\right)e^{-i\Omega t}\right) (-1)^{\hat{a}^{\dagger}\hat{a}}\times \\ & \times \hat{D}^{\dagger}\left(\mu\left(\mu\gamma-\nu\gamma^{*}\right)e^{i\Omega t}+\nu\left(\mu\gamma^{*}-\nu^{*}\gamma\right)e^{-i\Omega t}\right) \vert \psi(0)\rangle \\
   & = W_{0}\left(\mu\left(\mu\gamma-\nu\gamma^{*}\right)e^{i\Omega t}+\nu\left(\mu\gamma^{*}-\nu^{*}\gamma\right)e^{-i\Omega t}\right)
\end{align*}
con
\begin{align*}
  W_{0}(z) = \frac{2}{\pi} \langle \psi(0)\vert \hat{D}(z) (-1)^{\hat{a}^{\dagger}\hat{a}} \hat{D}^{\dagger}(z) \vert \psi(0) \rangle
\end{align*}
es decir, se tiene la función de Wigner independiente del tiempo para un estado coherente $\ket{z}$ que evoluciona en el tiempo, lo que nos permite utilizar la teoría de la función de Wigner independiente del tiempo. Por otra parte, del estado inicial se tiene
\begin{equation*}
  \vert \psi(0) \rangle = A \left(\vert \alpha \rangle + c \vert  \beta\rangle \right)
\end{equation*}
Sustituyendo el ket en la definición de la matriz de densidad, se obtiene
\begin{align*}
  W_0(z) = & \frac{2A}{\pi} \left[ \langle \alpha \vert + c^* \langle \beta \vert \right] \hat{D}(z) (-1)^{\hat{a}^\dagger \hat{a}} \hat{D}^\dagger(z) \left[ \vert\alpha\rangle + c\vert \beta \rangle \right]                             \\
  =        & \frac{2A}{\pi} \bigg[ \langle \alpha \vert \hat{D}(z) (-1)^{\hat{a}^\dagger \hat{a}} \hat{D}^\dagger(z) \vert\alpha\rangle + c^* \langle \beta \hat{D}(z) (-1)^{\hat{a}^\dagger \hat{a}} \hat{D}^\dagger(z) \vert\alpha\rangle \\ & + c \langle \alpha\vert \hat{D}(z) (-1)^{\hat{a}^\dagger \hat{a}} \hat{D}^\dagger(z) \vert\beta\rangle + |c|^2\langle \beta \vert \hat{D}(z) (-1)^{\hat{a}^\dagger \hat{a}} \hat{D}^\dagger(z) \vert\beta\rangle \bigg] \
\end{align*}
Por verificar que la función de Wigner es real, notemos que
\begin{align*}
  \left( c \langle \alpha\vert \hat{D}(z) (-1)^{\hat{a}^\dagger \hat{a}} \hat{D}^\dagger(z) \vert\beta\rangle \right)^* = & c* \langle \beta\vert \left( \hat{D}(z) (-1)^{\hat{a}^\dagger \hat{a}} \hat{D}^\dagger(z) \right)^* \vert\alpha\rangle
  \\ =& c^* \langle \beta \hat{D}(z) (-1)^{\hat{a}^\dagger \hat{a}} \hat{D}^\dagger(z) \vert\alpha\rangle
\end{align*}
La parte imaginaria de estos términos se anula en la suma dado que son complejos conjugados, y la función tiene dominio real. Calculando las funciones de Wigner en términos de estos estados coherentes
\begin{align*}
  W_0(z) = & \frac{2A}{\pi} \bigg[ e^{-2|\alpha - z|^2} + c^* \langle \beta \hat{D}(z) (-1)^{\hat{a}^\dagger \hat{a}} \hat{D}^\dagger(z) \vert\alpha\rangle                                                                                                                          \\ & + c \langle \alpha\vert \hat{D}(z) (-1)^{\hat{a}^\dagger \hat{a}} \hat{D}^\dagger(z) \vert\beta\rangle + |c|^2 e^{-2|\beta - z|^2} \bigg] \\ = & \frac{2A}{\pi} \bigg[ e^{-2|\alpha - z|^2} + c^* \langle 2z-\beta \vert \alpha \rangle \left[ e^{\frac{1}{2}(z^*(z-\beta) - (z-\beta)^* z)} e^{\frac{1}{2}(z^*\alpha - z \alpha^*)} e^{\frac{1}{2}(z^*\beta - z \beta^*)}\right]^* \\ & + c \langle \alpha \vert 2z - \beta\rangle e^{\frac{1}{2}(z^*(z-\beta) - (z-\beta)^* z)} e^{\frac{1}{2}(z^*\alpha - z \alpha^*)} e^{\frac{1}{2}(z^*\beta - z \beta^*)} + |c|^2 e^{-2|\beta - z|^2} \bigg] \\
  =        & \frac{2A}{\pi} \bigg\{ e^{-2|\alpha - z|^2} + c^* e^{(2z-\beta)^*\alpha - \frac{1}{2}|2z-\beta|^2 - \frac{1}{2}|\alpha|^2}  \left[ e^{\frac{1}{2}(z^*(z-\beta) - (z-\beta)^* z)} e^{\frac{1}{2}(z^*\alpha - z \alpha^*)} e^{\frac{1}{2}(z^*\beta - z \beta^*)}\right]^* \\ & + c e^{\alpha^*(2z-\beta) - \frac{1}{2}|2z-\beta|^2 - \frac{1}{2}|\alpha|^2} e^{\frac{1}{2}(z^*(z-\beta) - (z-\beta)^* z)} e^{\frac{1}{2}(z^*\alpha - z \alpha^*)} e^{\frac{1}{2}(z^*\beta - z \beta^*)} + |c|^2 e^{-2|\beta - z|^2} \bigg\} \\
  =        & \frac{2A}{\pi} \bigg\{ e^{-2|\alpha - z|^2} + c^* e^{(2z-\beta)^*\alpha - \frac{1}{2}|2z-\beta|^2 - \frac{1}{2}|\alpha|^2} e^{-\frac{1}{2}(z^*(z-\beta) - (z-\beta)^* z)} e^{-\frac{1}{2}(z^*\alpha - z \alpha^*)} e^{-\frac{1}{2}(z^*\beta - z \beta^*)}               \\ & + c e^{\alpha^*(2z-\beta) - \frac{1}{2}|2z-\beta|^2 - \frac{1}{2}|\alpha|^2} e^{\frac{1}{2}(z^*(z-\beta) - (z-\beta)^* z)} e^{\frac{1}{2}(z^*\alpha - z \alpha^*)} e^{\frac{1}{2}(z^*\beta - z \beta^*)} + |c|^2 e^{-2|\beta - z|^2} \bigg\}
\end{align*}
