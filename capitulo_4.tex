\chapter{Función de Wigner}

Es conocido que para sistemas microscópicos, la mecánica cuántica puede describirlos a través de un vector de estado $\ket{\psi}$ para estados puros o el operador de densidad $\hat{\rho}$ tanto para estados puros como mixtos. Existe otra representación que ayuda a exponer las propiedades del estado cuántico usando el espacio fase a través de la función de Wigner (1932)

La mecánica estadística nos permite comparar las similitudes y diferencias entre el sistema de una sola partícula en mecánica cuántica y un ensemble de partículas en mecánica clásica. Esto fue propuesto por Max Born en su conferencia Nobel (1954) usando como ejemplo el sistema cuántico de una partícula confinada a una caja.

De manera clásica, una distribución de probabilidad del espacio fase puede describir el estado de un sistema clásico, como en el caso de las partes real e imaginaria de la amplitud $\alpha$ del oscilador electromagnético. Una distribución $W(q,p)$ cuantifica la probabilidad de encontrar una partícula en la configuración $q$, $p$ en una medición simultánea. Conociendo esta distribución de probabilidad del espacio fase, todas las cantidades se pueden predecir por cálculo directo \cite{Leonhardt}.

Dado que $\hat{q}$ y $\hat{p}$ no son observables compatibles, el principio de incertidumbre no permite la medición simultánea y precisa de ambas cantidades, por lo que un espacio fase en el sentido clásico no se puede definir directamente como una distribución de probabilidad. Sin embargo, se puede definir un objeto que dependa de los eigenvalores de $\hat{x}$ y $\hat{p}$. Esta definición se realiza sacrificando que la función sea no negativa en todo punto. Se le denomina entonces una función de cuasi-probabilidad, y no corresponde a ninguna cantidad medible directamente. Estas funciones son útiles para reconstruir un estado cuántico de un sistema dado, y también pueden dar información sobre la fase del oscilador armónico a falta de un operador que pueda medir esta propiedad \cite{Moya}.

La clasificación del ordenamiento de estos operadores permite realizar cálculos más inmediatos a través del uso de propiedades que resultan de tal orden. Una función simétricamente ordenada o Weyl-ordenada se tiene cuando hay simetría respecto al ordenamiento de los operadores, por ejemplo ${\hat{a}^\dagger}^2 \hat{a} + \hat{a}^\dagger \hat{a}\hat{a}^\dagger + \hat{a} {\hat{a}^\dagger}^2 $ \cite{Mandel}. Una función de los operadores $\hat{a}$ y $\hat{a}^\dagger$ se dice normalmente ordenada cuando en cada producto los operadores de creación $\hat{a}^\dagger$ se multiplican por la derecha a los de aniquilación $\hat{a}$. Cuando se reescribe una función de estos operadores a través del uso de las relaciones de conmutación para seguir esta estructura se le llama ordenamiento. Por ejemplo la segunda potencia del operador número $\hat{N}={\hat{a}^\dagger}^2 \hat{a}^2 + \hat{a}^\dagger \hat{a}$. El anti-ordenamiento es el proceso contrario, de escribir las funciones de los operadores escalera de forma que el de aniquilación $\hat{a}$ preceda en cada término como factor al de creación $\hat{a}^\dagger$.

La función de Wigner, de manera análoga a las funciones de densidad de probabilidad clásicas, se define como la transformada de Fourier de una función característica $\chi(\beta)$
\begin{equation}
  W(\alpha) = \frac{1}{\pi^2}\int \exp{\left[ \beta^* \alpha - \beta \alpha^* \right]} \chi(\beta) d^2\beta.
\end{equation}
La función característica debe ser simétricamente ordenada. En este caso, el operador de densidad brinda la descripción completa del sistema, y está determinado por su función característica
\begin{equation}
  \chi(\alpha) = \text{Tr}\left\{ \rho \hat{D}(\alpha)  \right\}.
\end{equation}
A parte de la función de Wigner, existen otras funciones de distribución cuasi-probabilísticas como lo son la función $Q$ de Husimi o la función $P$ de Glauber-Sudarshan. La diferencia entre ellas radica en el orden de los factores de aniquilación y creación, usando una función característica ad hoc a su  ordenamiento. La función $P$ es conveniente para evaluar funciones normalmente ordenadas, mientras que la función $Q$ lo es para evaluar funciones anti-normalmente ordenadas.

La función de Wigner también se puede construir usando la perspectiva de funciones de onda como Wigner (1932) la describió originalmente a partir de la descripción de un salto de una partícula de una posición $q'$ a otra $q''$. Sea $\xi = q'' -q'$ la distancia de esta transición. En el caso de una dimensión describimos el estado de la partícula con el operador de densidad $\hat{\rho}$ y el elemento de matriz $\bra{q''}\hat{\rho}\ket{q'}$ describe el salto del eigenestado de posición $\ket{q'}$ al eigenestado de posición $\ket{q''}$. Sea $q = (q'-q'')/2$ el punto medio entre la posición inicial y final de la transición y expresamos $q'$ y $q''$ en términos de la distancia de transición y el punto medio
\begin{equation}
  q' = q - \frac{1}{2}\xi; \quad q'' = q + \frac{1}{2}\xi,
\end{equation}
y el elemento de matriz que describe la transición es $\bra{q + \frac{1}{2}\xi}\hat{\rho}\ket{q - \frac{1}{2}\xi}$. Dado que el momento es el generador de traslaciones, se asocia el momento $p$ de la partícula con el salto $\xi$ de la partícula. La transformada de Fourier del salto brinda la distribución del momento y define a la función de Wigner
\begin{equation}
  W(q,p) = \frac{1}{2\pi\hbar} \int_{-\infty}^{\infty} \text{d}\xi \exp{\left( -\frac{i}{\hbar}\xi p \right)} \bra{q + \frac{1}{2}\xi}\hat{\rho}\ket{q - \frac{1}{2}\xi}
\end{equation}
donde $1/(2\pi\hbar)$ es el factor de normalización. haciendo el cambio de variable $y = \xi/2$
\begin{align}
  W(q,p) & = \frac{1}{\pi\hbar} \int_{-\infty}^{\infty} \text{d}y \exp{\left( -2\frac{i}{\hbar}y p \right)} \bra{q + \frac{1}{2}\xi}\hat{\rho}\ket{q - \frac{1}{2}\xi}                                  \nonumber \\
         & = \frac{1}{\pi\hbar} \int_{-\infty}^{\infty} \text{d}y \exp{\left( -2\frac{i}{\hbar}y p \right)} \bra{-y}e^{-iq\hat{p}/\hbar}  \hat{\rho} e^{iq\hat{p}/\hbar} \ket{y}                        \nonumber \\
         & = \frac{1}{\pi\hbar} \int_{-\infty}^{\infty} \text{d}y \bra{-y} e^{ip\hat{q}/\hbar} e^{-iq\hat{p}/\hbar}  \hat{\rho} e^{iq\hat{p}} e^{-ip\hat{q}/\hbar} \ket{y}                              \nonumber \\
         & = \frac{1}{\pi\hbar} \int_{-\infty}^{\infty} \text{d}y \bra{y} (-1)^{\hat{a}^\dagger\hat{a}} e^{ip\hat{q}/\hbar} e^{-iq\hat{p}/\hbar}  \hat{\rho} e^{iq\hat{p}} e^{-ip\hat{q}/\hbar} \ket{y}
\end{align}
se expresa entonces la función de Wigner en términos de la traza del operador \cite{Moya}
\begin{equation}
  (-1)^{\hat{a}^\dagger\hat{a}} \hat{D}^{\dagger}(\alpha) \hat{\rho} \hat{D}(\alpha),
\end{equation}
como
\begin{equation}\label{eq:c4-moya}
  W(q,p) = \frac{1}{\pi} \text{Tr}{\left\{ (-1)^{\hat{a}^\dagger\hat{a}} \hat{D}^{\dagger}(\alpha) \hat{\rho} \hat{D}(\alpha) \right\}},
\end{equation}
donde por convención se toma $\hbar$ como la unidad. Para calcular la función de Wigner en términos de estados coherentes puros, se utiliza $\hat{\rho} = \bra{\beta}\ket{\beta}$ y se puede demostrar que \cite{sanchez}
\begin{equation*}
  W(\alpha) = \frac{1}{\pi}e^{-2|\beta-\alpha|}.
\end{equation*}