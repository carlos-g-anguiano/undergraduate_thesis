\chapter{Introducci\'on}

La mecánica cuántica (MC) es un modelo físico que describe el comportamiento de las formas más pequeñas de la materia a través de la cuantización de la energía. La razón de su existencia se da a partir de fenómenos físicos que la mecánica clásica falla en predecir. Algunos de estos fenómenos son la radiación de cuerpo negro, el efecto fotoeléctrico y el no colapso de los electrones en los átomos.

Un experimento que ilustra la necesidad de un espacio complejo de considerar un espacio vectorial complejo para modelar fenómenos físicos es el de Stern-Gerlach (SG). El experimento consiste en átomos de plata (Ag) disparados a lo largo de dos piezas polares, teniendo una de estas un borde afilado. De los 47 electrones del átomo de plata, uno (5s) no tiene una contraparte simétrica, y genera un momento angular intrínseco. El momento magnético es proporcional al espín, y estos átomos al pasar a lo largo del espacio entre las dos piezas polares experimentarán una fuerza dependiendo de la dirección de su componente $z$. Estos colisionan en una pantalla, donde clásicamente se esperaría una distribución Gaussiana de puntos de choque de los átomos a lo largo de la pantalla. Sin embargo en realidad se observan dos manchas bien distinguidas entre sí. Del experimento se concluye que son solo dos los posibles valores del espín del electrón, sucediendo con la misma probabilidad.

Realizando este experimento de manera secuencial, pero direccionando el campo magnético en la componente $x$ y bloqueamos con una pantalla el haz de átomos con electrones con $S_z^-$. De esta forma asumiríamos que la mitad de los átomos pasando el segundo experimento tendrían una configuración $S_z^+$, $S_x^-$ y la otra mitad
$S_z^+$, $S_x^+$. Para verificar esta hipótesis, se puede realizar un tercer experimento de SG en secuencia, de nueva cuenta generando un campo magnético no homogéneo a lo largo del eje $z$. Si todos los átomos tienen una configuración con $S_z^+$, se esperaría que todos los átomos se vean reflejados en la pantalla superior, sin embargo, este no es el caso. La "selección" de átomos en el segundo experimento de SG destruye cualquier información del primer experimento de SG en la componente $z$. En notación de Dirac, estas posibles combinaciones se pueden escribir como:
\begin{equation}
  \label{1.1}
  \ket{S_x; \pm} = \pm \frac{1}{\sqrt{2}}\ket{S_z; +} + \frac{1}{\sqrt{2}}\ket{S_z; -}\,.
\end{equation}
Esto se puede representar en un espacio vectorial bidimensional complejo. Esta necesidad surge de utilizar la misma base $\ket{S_z; \pm}$ para representar los estados de espín en $y$. El espacio se describe por combinaciones lineales de los vectores base y coeficientes complejos
\begin{equation}
  \label{1.2}
  \ket{S_y; \pm} = \frac{1}{\sqrt{2}}\ket{S_z; +} \pm \frac{i}{\sqrt{2}}\ket{S_z; -}\,.
\end{equation}

% Aquí, ^x y ^p obedecen la relación canónica de conmutación, por lo tanto su espectro no tiene máximo y es continuo (Ver pág. 40 Leonhardt)

El espacio vectorial complejo utilizado tiene una dimensionalidad que depende el sistema físico estudiado. En el caso del experimento de SG previamente considerado, la dimensión es dos. En casos donde las variables son continuas, como es el caso del momento y la posición, estos se describen por espacios de infinita dimensión denominados de Hilbert, cuyos elementos $\ket{\psi}$ se denominan estados, y a las variables se les representa por observables $\hat{A}$, en este caso $\hat{p}$ y $\hat{x}$ para el momento y posición, respectivamente. De la naturaleza de un espacio vectorial, se obtiene el principio de superposición, dados dos estados $\ket{\psi_1}$ y $\ket{\psi_2}$, cualquier combinación lineal $c_1 \ket{\psi_1} + c_2 \ket{\psi_2}$ con $|c_1|^2 + |c_2|^2 = 1$ es también un estado puro del sistema. El principio de superposición explica la interferencia y, de acuerdo a la interpretación probabilista de Max Born, los módulos cuadrados de los coeficientes denotan las probabilidades de que el sistema se encuentre en dicho estado al momento de realizar una medición.

Dos operadores se dicen compatibles si son tales que su conmutación es cero, es decir $[\hat{A}, \hat{B}] = \hat{A}\hat{B} - \hat{B}\hat{A} = 0$, es decir, que el orden de aplicación de los operadores afecta de igual manera a un estado $\hat{A} \hat{B} \ket{\psi} = \hat{B} \hat{A} \ket{\psi}$. Por otro lado, si la diferencia es distinta de cero, se dice que son no compatibles. Este último es el caso de los operadores $\hat{x}$ y momento $\hat{p}$, cuya relación de conmutación es
\begin{equation}
  \label{1.3}
  [\hat{x}, \hat{p}] = i\hbar \hat{I}\,.
\end{equation}

Dado un observable $\hat{A}$, se puede definir un operador $\Delta A \def \hat{A} - \langle \hat{A} \rangle$. El cuadrado del valor esperado de este operador se conoce como la dispersión $\langle (\Delta A)^2 \rangle = \langle \hat{A}^2\rangle  - \langle\hat{A}\rangle^2$. La dispersión se anula cuando el estado tratado es un eigenestado del observable $\hat{A}$. A partir de estas definiciones, para dos observables $\hat{A}$ y $\hat{B}$ y cualquier estado se debe satisfacer la relación de incertidumbre generalizada \cite{Sakurai} (Sakurai)

\begin{equation}
  \label{1.4}
  \langle (\Delta A)^2 \rangle \langle (\Delta B)^2 \rangle \geq \frac{1}{4} \left| \left\langle \left[ \hat{A}, \hat{B} \right] \right\rangle  \right|^2 \,.
\end{equation}