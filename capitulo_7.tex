\chapter{Estados Coherentes}
\section{Estados coherentes}

% To do - First talk about quadrature of the fields and phasor diagrams
El valor esperado de el operador de aniquilación $\hat{a}$ y los de cuadratura $\hat{X}$ y $\hat{Y}$ son cero, por lo que no tienen una amplitud bien definida de manera clásica. Por otro lado, el haz de un láser es coherente y tiene definida una amplitud y fase. Los estados que representan este fenómeno son los estados coherentes.

Los estados coherentes son el equivalente a un haz monocromático de luz en la mecánica cuántica en el sentido de que son estados de mínima incertidumbre.  Paquetes de onda coherentes derivados del OAC fueron estudiados primeramente por Shcrödinger en 1926, y estos inspiraron el trabajo de Roy J. Glauber que introdujo los estados coherentes (también llamados de Glauber) publicado en 1963 y por el cuál ganó el premio Nobel en 2005. La luz de un láser tiene una amplitud definida, y los estados coherentes $\alpha$ se definen como eigenestados del operador amplitud o aniquilación $\hat{a}$ (Leonhardt)
\begin{equation*}
\hat{a} \ket{\alpha} = \alpha\ket{\alpha}
\end{equation*}
donde $\alpha$ es en general un número complejo adimensional (Fox)

El significado físico de $\alpha$ se da considerando un modo polarizado de frecuencia angular $\omega$ dentro de una cavidad de volumen $V$. Se puede definir a $\alpha$ en términos de los valores esperados adimensionales de los operadores de cuadratura
\begin{equation*}
  \alpha = X^2 + iY^2
\end{equation*}
Se puede separar $\alpha$ en su amplitud $|\alpha|=\sqrt{X^2 + Y^2}$ y fase $\phi$ considerando $X = |\alpha|\cos\theta$ y $Y=|\alpha|\sen\theta$ como
\begin{equation*}
  \alpha = |\alpha|e^{i\phi}
\end{equation*}
el estado coherente entonces se puede representar de forma fasorial con longitud $|\alpha|$ y en un ángulo $\phi$.
(Figura de fasores)
Las cuadraturas están sujetas a la misma incertidumbre que la posición y el momento generalizado del oscilador armónico simple
\begin{equation*}
  \Delta X \Delta Y \geq \frac{1}{4}
\end{equation*}
dado que no hay preferencia entre las dos cuadraturas, se concluye que sus incertidumbres son iguales
\begin{equation*}
  \Delta X = \Delta Y = \frac{1}{2}
\end{equation*}
Por ser $\hat{a}$ no Hermitiano, los eigenvalores de los estados coherentes son complejos, y estos corresponden físicamente a las amplitudes de onda complejas en óptica clásica, con magnitud $|\alpha|$ y fase $\arg \alpha$. El estado cuántico de un haz láser continuo es un ensemble de estados coherentes con fase aleatoria.

El estado vacío es un estado coherente, ya que satisface la ecuación $\hat{a} \ket{0} = 0 \ket{0} = 0$, y corresponde a un estado de amplitud cero. La energía promedio de un estado coherente corresponde a
\begin{equation*}
\langle \alpha \vert \hat{H}\vert \alpha \rangle = \left\langle \alpha \right\vert \hat{a}^{\dagger}\hat{a} + \frac{1}{2} \left\vert \alpha \right\rangle = |\alpha|^2 + \frac{1}{2}
\end{equation*}

El valor esperado del operador de aniquilación (o también llamado amplitud) $\hat{a}$, y las cuadraturas, bajo los estados coherentes ahora ti tienen un amplitud definida.
\begin{equation*}
\langle \alpha \vert \hat{a} \vert\alpha \rangle = \alpha \quad \langle \alpha \vert \hat{X} \vert \alpha \rangle = \frac{\mathrm{Re}\alpha}{\sqrt{2}} \quad \langle \alpha \vert \hat{Y} \vert \alpha\rangle = \frac{\mathrm{Im}\alpha}{\sqrt{2}}
\end{equation*}
(Figura de distribución de fotones Agarwal)
Los estados coherentes se pueden descomponer en una suma de estados número
\begin{equation*}
  \ket{\alpha} = \sum_{n=0}^{\infty} c_n \ket{n}
\end{equation*}
Los coeficientes se calculan a partir de la relación de recursión
\begin{equation*}
  c_{n+1} = \frac{\alpha}{\sqrt{n+1}}c_n
\end{equation*}
y resultan
\begin{equation*}
  c_n = \frac{\alpha^n }{\sqrt{n!}}c_0
\end{equation*}
El valor de $c_0$ se puede encontrar a partir de la condición de normalización, como los estados número son ortonormales, se obtiene
\begin{align*}
  1 = \braket{\alpha}{\alpha} & = c_0^2 \sum_{n=0}^{\infty} \frac{(\alpha^n)^2}{(\sqrt{n!})^2}\braket{n}{n} \\ &= c_0^2  \sum_{n=0}^{\infty} \frac{(|\alpha|^2)^2}{n!}\braket{n}{n}\\ &= c_0^2 e^{|\alpha|^2} \\
  c_0                         & = e^{-\frac{1}{2}|\alpha|^2}
\end{align*}
y la superposición de estados número resulta de manera explícita
\begin{equation*}
  \ket{\alpha} = e^{-\frac{1}{2}|\alpha|^2} \sum_{n=0}^{\infty} \frac{\alpha^n}{\sqrt{n!}}\ket{n}
\end{equation*}
La probabilidad de encontrar $n$ fotones en estado coherente se da por una distribución de Poisson con media $|\alpha|^2$, y esta es a su vez igual a la varianza $\langle n \rangle = |\alpha|^2 = \langle n^2 \rangle = \langle n \rangle^2$.

Los estados coherentes forman una base sobrecompleta, es decir, son completos pero no son ortonormales entre sí. La condición de completez es
\begin{equation*}
  \frac{1}{\pi}\int d^2 \alpha \ket{\alpha}\bra{\alpha} = 1
\end{equation*}
donde $\alpha$ es un número complejo y la integral se realiza sobre el plano complejo, siendo $d^2\alpha = dxdy$, con $x = \mathrm{Re}\alpha$ y $y = \mathrm{Im}\alpha$. La desmostración se da sustituyendo la definición de $\ket{\alpha}$ en la condición de completez y verificando que $\sum_n \ket{\alpha}\bra{\alpha}=1$. sustituyendo en el lado izquierdo
\begin{equation*}
  \sum_{n.m} \frac{1}{\sqrt{n!m!}}\int d^2\alpha (\alpha)^n \alpha^{*m}\ket{n}\bra{m}e^{-|\alpha|^2}
\end{equation*}
que cambiando a coordenadas polares
\begin{align*}
  1 & = \sum_{n,m}\frac{1}{\sqrt{n!m!}}\int_0^{\infty} rdr \int_{0}^{2\pi}d\theta (r)^{n+m}e^{i\theta(n-m)}\ket{n}\bra{m}e^{-r^2} \\ & = \sum_{n,m}\frac{1}{\sqrt{n!m!}}\int_0^{\infty} rdr  (r)^{n+m} \delta_{nm} \ket{n}\bra{m}e^{-r^2} \\
    & = \sum_{n}\frac{1}{n!}\int_0^{\infty} dr  r^{2n+1} \ket{n}\bra{n}e^{-r^2}                                                   \\
    & = \sum_{n} \ket{n}\bra{n} \frac{1}{n!}\int_0^{\infty} dr  r^{2(n+1)-1} e^{-r^2}                                             \\
    & = \sum_{n} \ket{n}\bra{n} \frac{1}{n!} \Gamma(n+1)                                                                          \\
  1 & = \sum_{n} \ket{n}\bra{n}
\end{align*}
La sobrecompletitud hace que los estados sean no ortogonales entre si, el producto interno entre dos elementos del espacio de estados coherentes es distinto de cero, y es en particular se puede calcular de la expresión de estado coherente
\begin{align*}
  \braket{\alpha}{\beta} & = e^{-\frac{1}{2}|\alpha|^2} e^{-\frac{1}{2}|\beta|^2} \sum_{n,m=0}^{\infty} \frac{{\alpha^*}^n}{\sqrt{n!}}\frac{\beta^m}{\sqrt{m!}} \braket{n}{m} \\
                         & = e^{-\frac{1}{2}|\alpha|^2} e^{-\frac{1}{2}|\beta|^2} \sum_{n=0}^{\infty} \frac{{(\alpha^*\beta)}^n}{n!} \braket{n}{n}                            \\
                         & = \exp{\left( - \frac{1}{2}|\alpha|^2 - \frac{1}{2}|\beta|^2 + \alpha^*\beta \right)}
\end{align*}
De lo anterior, un estado coherente se puede escribir en términos de otros estados coherentes. Considerando el braket de un par de estados coherentes $\ket{\alpha}$ y $\ket{\gamma}$, y utilizando la condición de completez
\begin{align*}
  \braket{\alpha}{\gamma}             & =  \bra{{\alpha}}\left(\frac{1}{\pi}\int d^2 \alpha \ket{\beta}\bra{\beta} \right) \ket{\gamma}      \\
                                      & = \left(\frac{1}{\pi}\int d^2 \beta \braket{\alpha}{\beta}\braket{\beta}{\gamma} \right)             \\
  \frac{\braket{\alpha}{\gamma}}{\pi} & = \left(\int d^2 \beta \frac{\braket{\alpha}{\beta}}{\pi} \frac{\braket{\beta}{\gamma}}{\pi} \right)
\end{align*}
Escrito de esta forma, podemos ver que $\frac{\braket{\alpha}{\beta}}{\pi}$ es un núcleo reproductor denotado por $K(\alpha, \beta)$, que proviene de la no-ortogonalidad de los estados coherentes. Se puede entonces reescribir lo anterior como
\begin{equation*}
  K(\alpha, \gamma) = \int d^2 \beta K(\alpha, \beta) K(\beta, \gamma)
\end{equation*}
Así, de nuevo por la condición de completez, podemos escribir cualquier estado coherente $\ket{\alpha}$ en términos de otro $\ket{\beta}$
\begin{equation*}
  \ket{\alpha} = \int K(\alpha, \beta) \ket{\beta} d^2\beta
\end{equation*}

