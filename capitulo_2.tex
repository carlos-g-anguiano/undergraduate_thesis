\chapter{Cuantizaci\'on del campo electromagn\'etico}

\section{Ecuaciones de Maxwell}

Las variables de campo eléctrico $\mathbf{E}$, y magnético $\mathbf{H}$, así como el de desplazamiento $\mathbf{D}$ el de inducción magnética $\mathbf{B}$ se pueden tratar como observables, cuyos valores esperados son el promedio de un ensemble cuántico de campos, es decir, se pueden representar por operadores Hermitianos $\hat{\mathbf{E}}$, $\hat{\mathbf{H}}$, $\hat{\mathbf{D}}$ y $\hat{\mathbf{B}}$ respectivamente. La linealidad de las ecuaciones de Maxwell permiten que obedezcan la interpretación estadística de Max Born, viendo que el promedio de cada una de las leyes se reduce a las leyes de Maxwell aplicadas a los promedios. Por el mismo argumento de linealidad, las ecuaciones constitutivas son válidas para estos mismos operadores, y se tiene (en SI) $\mathbf{D} = \epsilon_0 \epsilon \mathbf{E}$ y $\mathbf{B} = \mu_0 \mu \mathbf{H}$ con $\epsilon_0 \mu_0 = c^{-2}$.
Las ecuaciones de Maxwell rigen el comportamiento de los campos electromagnéticos clásicos y cuánticos.
\begin{align}
  \nabla \cdot \hat{\mathbf{B}}  & = 0 \,,                                              \label{EM.1}                 \\
  \nabla \times \hat{\mathbf{E}} & = - \frac{1}{c} \frac{\partial \hat{\mathbf{B}}}{\partial t}\,,      \label{EM.2} \\
  \nabla \cdot \hat{\mathbf{D}}  & = 0 \,,                                        \label{EM.3}                       \\
  \nabla \times \hat{\mathbf{B}} & = \frac{1}{c}
  \frac{\partial \hat{\mathbf{E}}}{\partial t}\,, \label{EM.4}
\end{align}
donde todos los campos son dependientes de la posición $\mathbf{r}$ y del tiempo $t$\footnote{Se utilizan unidades de CGS a lo largo de este desarrollo.}

Se define el operador de campo vectorial $\hat{\mathbf{A}}$ de manera que cumpla
\begin{align}
  \hat{\mathbf{E}} & = -\frac{1}{c}\frac{\partial \hat{\mathbf{A}}}{\partial t}, \label{EM.5} \\
  \hat{\mathbf{B}} & = \nabla \times \hat{\mathbf{A}}\,.\label{EM.6}
\end{align}
% Justificación de la elección del Gauge de Coulomb
Se elige entonces el gauge de Coulomb, haciendo el potencial escalar cero y el potencial vectorial $\hat{\mathbf{A}}$ tal que satisfaga la condición de transversalidad $\nabla \cdot \hat{\mathbf{A}} = 0$ \footnote{El presente desarrollo se realiza en el dominio no relativista \cite{Agarwal_2012}.}, haciendo que la primera ecuación de Maxwell (\ref{EM.4}) se cumpla de manera directa. Sustituyendo las ecuaciones (\ref{EM.5}) y (\ref{EM.6})
en la ecuaci\'on de Ampere-Maxwell (\ref{EM.4})
\begin{align}
  \label{EM.7}
  \nabla \times \hat{\mathbf{B}} = \nabla \times \left(\nabla\times\hat{\mathbf{A}}\right)
  =\frac{1}{c}\frac{\partial \hat{\mathbf{E}}}{\partial t} = \frac{1}{c}\frac{\partial}{\partial t} \left(-\frac{1}{c}\frac{\partial\hat{\mathbf{A}}}{\partial t}\right) = -\frac{1}{c^{2}}\frac{\partial^{2}\hat{\mathbf{A}}}{\partial t^{2}} \,,
\end{align}
obteniendo la ecuación de onda electromagnética
\begin{equation}
  \label{EM.8}
  \nabla^2 \hat{\mathbf{A}} = \frac{1}{c^2} \frac{\partial^2 \hat{\mathbf{A}}}{\partial t^2}\,,
\end{equation}
% Demostrar que los operadores de campo electromagnético son Hermitianos
la conocida solución particular de la onda plana se puede obtener por el método de separación de variables y se reduce al resultado de la onda plana
\begin{equation}
  \label{EM.9}
  \hat{\mathbf{A}}_k = \hat{c}_k \mathbf{e}^{(\lambda)}\exp{(i\mathbf{k}\cdot \mathbf{r})}\exp{(-i\omega_k t)} = \hat{c}_{k}\mathbf{u}_{k\lambda}(\mathbf{r}) \exp{(-i\omega_k t)} \,,
\end{equation}
con $\mathbf{k} = (k_x, k_y, k_z)$ el vector de número de onda, cuya magnitud está dada por $k = 2\pi/\lambda$ \cite{Riley}, y $\hat{\mathbf{u}}_{k\lambda}(\mathbf{r}) = \hat{\mathbf{e}}^{(\lambda)} \exp{(i\mathbf{k}\cdot \mathbf{r})}$ con $\hat{\mathbf{e}}^{(\lambda)}$ el vector de polarización, donde $\lambda=1,2$ es el índice de polarización \cite{Walls}.

De la teoría de Sturm-Liouville, al definir condiciones de frontera para nuestras coordenadas espaciales, se define un espacio de Hilbert cuyos elementos $\mathbf{u}(\mathbf{r})$ son funciones diferenciables y que satisfacen anularse en las fronteras dadas. La ecuación de eigenvalores se escribe como \cite{Arfken}:
\begin{equation}
  \label{EM.10}
  \mathcal{L} \hat{\mathbf{A}} = \left( \frac{\partial^2}{\partial x^2} + \frac{\partial^2}{\partial y^2} + \frac{\partial^2}{\partial z^2} \right) \hat{\mathbf{A}} = (k_x^2 + k_y^2 + k_z^2) \hat{\mathbf{A}} \,,
\end{equation}
con
\begin{equation}
  \label{EM.11}
  \mathcal{L} = \frac{\partial^2}{\partial x^2} + \frac{\partial^2}{\partial y^2} + \frac{\partial^2}{\partial z^2} \,.
\end{equation}
La solución general se puede obtener a partir de una superposición de las $k$-ésimas soluciones particulares. A su vez, esta superposición de soluciones se puede separar en dos sumandos de acuerdo a la magnitud de las frecuencias angulares \cite{Riley}. El campo electromagnético se describe entonces restringido a un volumen cúbico en el espacio y el potencial vectorial se expande en términos del conjunto discreto las funciones de modo ortogonales del espacio de Hilbert definido por el \textit{ansatz} (\ref{EM.9}). Separando los términos en las amplitudes incidentes (amplitudes que varían de acuerdo a $e^{-i\omega_k t}$ con $\omega>0$) y reflejadas (amplitudes que varían como $e^{i\omega_k t }$ y $\mathbf{A}^{(-)} = (\mathbf{A}^{(+)})^*$) % Usando la terminología de Griffiths Quantum Mechanics 3rd ed para la solución del pozo finito 
\begin{equation}
  \label{EM.12}
  \mathbf{A} = \mathbf{A}^{(+)} + \mathbf{A}^{(-)} = \sum_{k,\lambda} \left( \hat{c}_{k} \mathbf{u}_{k\lambda} (\mathbf{r})e^{-i\omega_k t} +  \hat{c}_{k}^* \mathbf{u}_{k\lambda}^* (\mathbf{r})e^{i\omega_{k} t} \right) \,,
\end{equation}
nos enfocaremos solo en los términos de amplitudes incidentes, las cuales deben de cumplir, además de ser solución a la ecuación de onda, con la condición de transversalidad
\begin{equation}
  \label{EM.13}
  \nabla \cdot \mathbf{u}(\mathbf{r}) = 0 \,,
\end{equation}
lo cual se cumple al momento de asignarle un vector de polarización $\mathbf{\hat{e}}^{(\lambda)}$ perpendicular a $\mathbf{k}$. A su vez las funciones forman un conjunto ortogonal bajo el producto interno del espacio de Hilbert. Se establecen condiciones de frontera periódicas que corresponden a modos de vibración que llevan a ondas estacionarias en un volúmen cúbico de paredes reflejantes de lados de longitud $L$. De esta condición de ortogonalidad
\begin{equation}
  \int_V \exp{(i\mathbf{k}\cdot\mathbf{r})} \exp{(-i\mathbf{k}\cdot\mathbf{r})} d\mathbf{r}
  =  \int_V d\mathbf{r}
  =  L^3 = V \,,
\end{equation}
\begin{equation}
  \label{EM.14}
  \int_V \mathbf{u}^{*}_k(\mathbf{r}) \mathbf{u}_{k'}(\mathbf{r}) d\mathbf{r} = L^3\delta_{kk'} \,.
\end{equation}
Así, las funciones de onda se escriben como
\begin{equation}
  \label{EM.15}
  \mathbf{u}_{k\lambda}(\mathbf{r}) = \mathbf{\hat{e}}_{k\lambda} \exp\left({i\mathbf{k}\cdot \mathbf{r}}\right) \,.
\end{equation}
El vector de propagación $\mathbf{k}$ en coordenadas cartesianas toma los valores por entrada
\begin{equation}
  \label{EM.16}
  k_i = \frac{2\pi n_i}{L}, \quad \text{con } i=x,y,z , \quad n_i = 0, \pm1, \pm 2, \dots
\end{equation}
La densidad de energía del campo electromagnético está dada por $u = \mathbf{E} \cdot \mathbf{D} + \mathbf{B} \cdot \mathbf{E}$ en SI. Considerando el vacío como medio y ningúna fuente, se puede reescribir como $u = \mathbf{E}^2 + \mathbf{B}^2$ (SI). Transformando a unidades CGS, se tiene $\mathbf{E} \to \mathbf{E}/\sqrt{4\pi \varepsilon_0}$ y $\mathbf{B} \to \sqrt{\frac{\mu_0}{4\pi}}\mathbf{B}$, y como $\mu_0=1$ y $\epsilon_0 = 1$ en CGS, la energía electromagnética dentro de un volumen es
\begin{equation}
  \label{EM.17}
  U = \frac{1}{8\pi}\int_V \left[ \mathbf{E}^2(\mathbf{r}, t) + \mathbf{B}^2(\mathbf{r}, t) \right] d\mathbf{r} \,.
\end{equation}
y en forma de operador se expresa con el Hamiltoniano
\begin{equation}
  \label{EM.17H}
  \hat{H} = \frac{1}{8\pi}\int_V \left[ \hat{\mathbf{E}}^2(\mathbf{r}, t) + \hat{\mathbf{B}}^2(\mathbf{r}, t) \right] d\mathbf{r} \,.
\end{equation}
El campo magnético y la inducción magnética se calculan en términos del conjunto de funciones ortonormales $\mathbf{u}_k(\mathbf{r})$ a partir del campo vectorial $\hat{\mathbf{A}}(\mathbf{r}, t)$
\begin{align}
  \hat{\mathbf{E}}           & = -\frac{1}{c} \frac{\partial \hat{\mathbf{A}}}{\partial t}                                                                                                                                                         \nonumber \\
                             & =-\frac{1}{c}\left[ \frac{\partial \hat{\mathbf{A}}^{(+)}}{\partial t} + \frac{\partial \mathbf{A}^{(-)}}{\partial t} \right]                                                                                       \nonumber \\
                             & = -\frac{1}{c} \sum_{k,\lambda} \left[ -i\omega_{k}\hat{c}_{k} \mathbf{u}_{k\lambda} (\mathbf{r})e^{-i\omega_k t} +  i\omega_{k}\hat{c}_{k}^* \mathbf{u}_{k\lambda}^* (\mathbf{r})e^{i\omega_{k} t} \right] \,,     \nonumber \\
  c^2 \vert\mathbf{E}\vert^2 & = c^2 \mathbf{E} \cdot \mathbf{E}                                                                                                                                                                                   \nonumber \\
                             & = \sum_{k,\lambda}\sum_{k',\lambda'}\left(-i\omega_{k}\hat{c}_{k}\mathbf{u}_{k\lambda} (\mathbf{r})e^{-i\omega_{k}t} + i \omega_{k} c^{*}_{k}\mathbf{u}^{*}_{k\lambda}(\mathbf{r})e^{i\omega_{k}t} \right)\cdot     \nonumber \\
                             & \cdot \left(-i\omega_{k'}\hat{c}_{k'}\mathbf{u}_{k'\lambda} (\mathbf{r})e^{-i\omega_{k'}t} + i \omega_{k'} \hat{c}^{*}_{k}\mathbf{u}^{*}_{k'\lambda'}(\mathbf{r})e^{i\omega_{k'}t} \right)                          \nonumber \\
                             & = \sum_{k,\lambda} \Big[ \left(-i\omega_{k}\hat{c}_{k} \mathbf{u}_{k\lambda} (\mathbf{r})e^{-i\omega_k t}\right)\cdot \left( i\omega_{k}\hat{c}_{k}^* \mathbf{u}_{k\lambda}^* (\mathbf{r})e^{i\omega_{k} t} \right) \nonumber \\
                             & +  \left(i\omega_{k}\hat{c}_{k}^* \mathbf{u}_{k\lambda} (\mathbf{r})e^{i\omega_{k} t}\right)\cdot \left(-i\omega_{k}\hat{c}_{k} \mathbf{u}_{k\lambda}^* (\mathbf{r})e^{-i\omega_k t}\right) \Big],
\end{align}
entonces
\begin{equation}
  \label{EM.18}
  \hat{\mathbf{E}}^2 = \frac{2}{c^2}\sum_{k,\lambda} \omega_{k}^2  |\mathbf{u}(\mathbf{r})|^2 \left( \hat{c}_{k} \hat{c}_{k}^* + \hat{c}_{k}^* \hat{c}_{k} \right) \,.
\end{equation}
Para el campo magnético, se selecciona la dirección de polarización del campo eléctrico, analizando solamente su carácter vectorial
$\mathbf{e}^{\lambda}= \mathbf{\hat{\textnormal{\bfseries\i}}}$, entonces
\begin{equation}
  \mathbf{B} = \mathbf{\hat{\textnormal{\bfseries\j}}} \left( \frac{\partial}{\partial z} A_x - \frac{\partial}{\partial x} A_z \right)
\end{equation}
pero como la dirección de polarización es en $\mathbf{\hat{\textnormal{\bfseries\i}}}$, solo la componente de $A_x$ es distinta de cero y
\begin{equation}
  \mathbf{B} = \mathbf{\hat{\textnormal{\bfseries\j}}} \frac{\partial}{\partial z} A_x
\end{equation}
como
\begin{equation}
  \frac{\partial \mathbf{u}_{k\lambda}(\mathbf{r})}{\partial z} = i k_z \mathbf{u}_{k\lambda}(\mathbf{r})
\end{equation}
y
\begin{equation}
  \frac{\partial \mathbf{u}^{*}_{k\lambda}(\mathbf{r})}{\partial z} = -i k_z \mathbf{u}^{*}_{k\lambda}(\mathbf{r}).
\end{equation}
\begin{equation}
  \hat{\mathbf{B}} = \mathbf{\hat{\textnormal{\bfseries\j}}} \sum_{k,\lambda} \left( i k_z \hat{c}_{k} \mathbf{u}_{k\lambda}(\mathbf{r})e^{-i\omega_{k} t} - i k_z \hat{c}_{k}^* \mathbf{u}_{k\lambda}^*(\mathbf{r})e^{i\omega_{k} t}  \right)
\end{equation}
Calculando su norma al cuadrado usando el producto punto para complejos
\begin{align}
  \hat{\mathbf{B}}^2 & = \sum_{k,\lambda} \Big[ \left(ik_z \hat{c}_{k} \mathbf{u}_{k\lambda}(\mathbf{r})e^{-i\omega_{k} t}\right)\left(-ik_z \hat{c}_{k}^{*} \mathbf{u}_{k\lambda}^*(\mathbf{r})e^{i\omega_{k} t} \right) \nonumber \\
                     & + \left(-ik_z \hat{c}_{k}^{*} \mathbf{u}_{k\lambda}^*(\mathbf{r})e^{i\omega_{k} t} \right)  \left(ik_z \hat{c}_{k} \mathbf{u}_{k\lambda}(\mathbf{r})e^{-i\omega_{k} t}\right) \Big]                \nonumber \\
                     & = 2 \sum_{k,\lambda} k_z^2 |\mathbf{u}(\mathbf{r})|^2 \left( \hat{c}_{k} \hat{c}_{k}^* + \hat{c}_{k}^* \hat{c}_{k} \right) \,.
\end{align}
Utilizando los operadores de cuadratura $\hat{X}$ y $\hat{Y}$ previamente definidos, se puede reescribir el campo eléctrico. Usando $\mathbf{u}_{k\lambda}(\mathbf{r})e^{-i\omega_k t} = \mathbf{e}_{k\lambda} e^{\mathbf{k}\cdot\mathbf{r}-i\omega_k t}$ se obtiene
\begin{equation}
  \mathbf{E} = i \sum_{k,\lambda} \left( \frac{8\hbar \pi \omega_k}{V} \right)^{1/2} \mathbf{e}_{k\lambda} \left[ \hat{X}\cos(\mathbf{k}\cdot\mathbf{r} - \omega t) + \hat{Y}\sen(\mathbf{k}\cdot\mathbf{r} - \omega t) \right]
\end{equation}
Las variables canónicas $\hat{X}$ y $\hat{Y}$ se pueden interpretar como amplitudes de las cuadraturas en las que se descompone el campo eléctrico. \cite{Mandel}
En nuestro eje coordenado, la propagación de la onda electromagnética se de en la dirección $\mathbf{\hat{\textnormal{\bfseries k}}}$, por lo que $|\mathbf{k}| = k = k_z$. Además $\omega_k = kc$ y se obtiene
\begin{equation}
  \label{EM.19}
  \hat{\mathbf{B}}^2 = 2 \sum_{k,\lambda} \frac{\omega_k^2}{c^2}  |\mathbf{u}(\mathbf{r})|^2 \left( \hat{c}_{k} \hat{c}_{k}^* + \hat{c}_{k}^* \hat{c}_{k} \right) \,,
\end{equation}
sustituyendo en la expresión del Hamiltoniano (\ref{EM.17H})
\begin{align}
  \hat{H} & = \frac{1}{8\pi} \Bigg[ \int_V \left( 2\sum_{k,\lambda} \frac{\omega_{k\lambda}^2}{c^2}  |\mathbf{u}_{k\lambda}(\mathbf{r})|^2 \left( \hat{c}_{k} \hat{c}_{k}^* + \hat{c}_{k}^* \hat{c}_{k} \right) \right) d\mathbf{r} \nonumber \\ &+ \int_V \left( 2 \sum_{k,\lambda} \frac{\omega_{k\lambda}^2}{c^2} |\mathbf{u}_{k\lambda}(\mathbf{r})|^2 \left( \hat{c}_{k} \hat{c}_{k}^* + \hat{c}_{k}^* \hat{c}_{k} \right) \right)d\mathbf{r} \Bigg] \nonumber \\
          & = \frac{1}{2\pi c^2} \sum_{k,\lambda} \omega_{k\lambda}^2 \left( \hat{c}_{k} \hat{c}_{k}^* + \hat{c}_{k}^* \hat{c}_{k} \right) \int_V |\mathbf{u}_{k\lambda}^2(\mathbf{r})|^2 d\mathbf{r} \,,
\end{align}
donde por la condición de ortogonalidad
\begin{equation}
  \label{EM.20}
  \hat{H} = \frac{V}{2\pi c^2}\sum_{k,\lambda} \omega_{k\lambda}^2 \left( \hat{c}_{k} \hat{c}_{k}^* + \hat{c}_{k}^* \hat{c}_{k} \right) \,,
\end{equation}
ahora si consideramos
\begin{equation}
  \label{EM.21}
  \hat{c}_{k} = \sqrt{\frac{\pi c^{2}\hbar}{\omega_{k}V}} \hat{a}_{k}\,,
\end{equation}
la ecuaci\'on (\ref{EM.20}) se convierte en:
\begin{align}
  \hat{H} & = \frac{V}{2\pi c^2}\sum_{k} \omega_{k}^2 \frac{\pi c^{2}\hbar}{\omega_{k}V}\left(\hat{a}_{k}\hat{a}^{*}_{k} + \hat{a}_{k}^{*} \hat{a}_{k} \right) = \frac{V}{2\pi c^2}\sum_{k} \omega_{k}^2 \frac{\pi c^{2}\hbar}{\omega_{k}V}\left(\hat{a}_{k}\hat{a}^{*}_{k} + \hat{a}_{k}^{*} \hat{a}_{k} \right) \\
\end{align}
\begin{equation}
  \label{EM.22}
  \hat{H} = \sum_{k} \frac{\hbar\omega_{k}}{2}\left(\hat{a}_{k}\hat{a}^{*}_{k} + \hat{a}_{k}^{*} \hat{a}_{k} \right)
\end{equation}
En orden de cuantizar el campo electromagnético, se escogen $\hat{a}_{k}$ y $\hat{a}^{*}_{k}$ como los operadores de aniquilaci\'on $\hat{a}_{k}$ y creaci\'on $\hat{a}^{\dagger}_{k}$ para cada modo del campo $k$. Como el objeto de estudio son fotones, el conmutador de bosones es el que se debe seguir
\begin{equation}
  [\hat{a}_k, \hat{a}^{\dagger}_{k'}] = [\hat{a}^{\dagger}_k, \hat{a}_{k'}] = 0, \quad [\hat{a}_k, \hat{a}_{k'}^{\dagger}] = \delta_{kk'} \,.
\end{equation}
\iffalse
  Los operadores de campo electromagnético $\mathbf{E}$, $\mathbf{B}$ y $\vec{A}$ son Hermitianos (demostrar, Agarwal), por lo que se pueden descomponer al campo $\mathbf{A}$ en la suma del campo $\mathbf{A}^{(+)}$ que contiene todas las frecuencias positivas  y su complejo conjugado $\mathbf{A}^{(-)}$. (Walls)

  Considere primero una onda electromagnética, descompuesta en dos términos de esta forma
  \begin{equation}
    \mathbf{E}^{(+)} = \hat{\varepsilon} E_0 \exp{\left(i\vec{k}\cdot \vec{r} - i\omega_k t\right)}, \omega > 0
  \end{equation}
  Donde $k = |\vec{k}| = \omega_k/c$y $\hat{\varepsilon}$ es el vector de polarización del campo electromagnético, el cual es ortogonal a la dirección de propagación y es perpendicular a la dirección de propagación $\vec{k}$. Hay dos direcciones de polarización, y se denota por el índice $s$ con valores $1$, $2$. De esta forma se tiene que el vector de polarización depende de dos índices y $\vec{\varepsilon} = \vec{\varepsilon}_{\vec{k},s}$ En términos del operador campo vectorial $\mathbf{A}$ del gauge de Coulomb
  \begin{align}
    \mathbf{A}^{(+)} & = -\frac{1}{c}\frac{\partial \mathbf{A}}{dt} \mathbf{E}^{(+)}                                                           \\
                     & = -\frac{(-i\omega_k)}{c}\hat{\varepsilon} E_0 \exp{\left(i\vec{k}\cdot \vec{r} - i\omega_k t\right)}, \quad \omega > 0 \\
                     & = \hat{\varepsilon} A_0 \exp{\left(i\vec{k}\cdot \vec{r} - i\omega_k t\right)}, \quad \omega > 0
  \end{align}
  Donde $A_0=\frac{(i\omega_k)}{c}E_0$. Esto nos permite construir la solución de onda plana de la ecuación de onda del campo vectorial $\mathbf{A}$. (Agarwal) Al término $\hat{\varepsilon} \exp{\left(i\vec{k}\cdot \vec{r}\right)}$ se le denomina función de modo y se le denotará por $\hat{u} = \hat{u}(\vec{r})$, satisfacen la ecuación de onda y de transversalidad al igual que $\mathbf{A}$. (Walls).

  Se establecen ahora condiciones de frontera, considerando un volumen cúbico de lados de longitud $L$ y periodicidad. Estas condiciones corresponden a modos de ondas viajeras que en paredes reflejantes se pueden describir como ondas estacionarias. Las funciones de modo forman un conjunto ortonormal completo con estas condiciones, es decir
  \begin{equation}
    \int_V \hat{u}_k^{*}(\vec{r}) u_{k'} \vec{r} d^3 r = \delta_{kk'}
  \end{equation}
  En coordenadas cartesianas, de las condiciones de frontera establecen los valores de $k_\alpha$ con $\alpha = x,y,z$.
  \begin{equation}
    k_i = \frac{2\pi n_i}{L}, \quad n_i = 0, \pm1, \pm2, \dots
  \end{equation}
  La suma en las direcciones de propagación $\vec{k}$ todos los modos y las dos direcciones de polarización $s$, denotando $V = L^3$, se tiene
  \begin{align}
    \mathbf{A}^{(+)} & = \sum_{\vec{k}, s} \frac{A_{\vec{k}s}}{\sqrt{V}} \hat{\varepsilon}_{\vec{k}s} \exp{ \left( i\vec{k}\cdot\vec{r} - i \omega_k t \right) } \\
                     & = \sum_{\vec{k},s} \frac{A_{\vec{k}s}}{\sqrt{V}} \hat{u} \exp{\left( - i \omega_k t \right)}
  \end{align}
  Así, el potencial vectorial es
  \begin{equation}
    \mathbf{A}(\vec{r}, t) = \sum_{\vec{k}, s} \left( \frac{\hbar}{2\omega_k \epsilon_0} \right)^{1/2} \left( \hat{a}^{\dagger} \hat{u}_k(\vec{r}) e^{-i\omega_k} t + \hat{a} \hat{u}_k(\vec{r})^{*} e^{i\omega_k} t \right)
  \end{equation}
  y el campo eléctrico es entonces
  \begin{equation}
    \mathbf{E}(\vec{r}, t) = i\sum_{\vec{k}, s} \left( \frac{\hbar\omega_k}{2\epsilon_0} \right)^{1/2} \left( \hat{a}^{\dagger} \hat{u}_k(\vec{r}) e^{-i\omega_k} t - \hat{a} \hat{u}_k(\vec{r})^{*} e^{i\omega_k} t \right)
  \end{equation}
  Esto iría en otro capítulo?
  Los coeficientes adminesionales de Fourier son números complejos. En orden de cuantizar el campo electromagnético, se escogen $\hat{a}^{\dagger}$ y $\hat{a}$ para ser adjunto uno al otro. Como el objeto de estudio son fotones, el conmutador de bosones es el que se debe seguir
  \begin{equation}
    [\hat{a}_k, \hat{a}^{\dagger}_{k'}] = [\hat{a}^{\dagger}_k, \hat{a}_{k'}] = 0, \quad [\hat{a}^{\dagger}_k, \hat{a}_{k'}] = \delta_{kk'}
  \end{equation}
\fi
\section{Cuantización del campo electromagnético}

La cuantización del campo electromagnético se da cuando se asocia a cada modo de luz $\mathbf{k},\lambda$ de la construcción realizada en la sección 2 a un oscilador armónico. Los modos de luz siguen las mismas relaciones de los operadores de creación $\hat{a}^{\dagger}$ y destrucción $\hat{a}$, que ahora corresponden a la creación o aniquilación de un fotón de energía $\hbar\omega$ en el modo $\mathbf{k}\lambda$. (Loudon, alterar un poco, está muy textual). Bajo este contexto, a los estado $\ket{\hat{n}_{\mathbf{k}\lambda}}$ se les denomina estados número del fotón o estados Fock del campo electromagnético.

Comparando las expresiones de energía del campo electromagnético y del Hamiltoniano del oscilador armónico, se obtiene la siguiente correspondencia entre los operadores de creación y aniquilación y los coeficientes complejos de la base ortogonal $\mathbf{u}_{k,\lambda}(\mathbf{r})$ \cite{Loudon}
\begin{equation}
  \left( \frac{2\hbar \pi c^2}{V\omega_k} \right)^{1/2} \hat{a}_{k\lambda} \leftrightarrow \hat{c}_{k}; \quad \left( \frac{2\hbar \pi c^2}{V\omega_k} \right)^{1/2} \hat{a}^{\dagger}_{k\lambda} \leftrightarrow \hat{c}_{k}^*.
\end{equation}
La luz es un objeto cuántico con estado $\ket{\Psi}$. Usando la imagen de Heisenberg, donde los operadores evolucionan en el tiempo y el estado es estacionario, se considera a los campos clásicos como los valores esperados de los observables $\mathbf{E}$, $\mathbf{D}$, $\mathbf{B}$ y $\mathbf{H}$ que actúan sobre los estados de la luz. la linealidad de las ecuaciones de Maxwell  permite que estas sean válidas para estos operadores. \cite{Leonhardt} De esta forma, se cuantiza la energía del campo electromagnético, remplazando $U$ por el operador Hamiltoniano
\begin{equation}
  \hat{H}  =  \frac{V}{2\pi c^2}\sum_{k,\lambda} \left( \frac{2\hbar \pi c^2}{V\omega_k} \right) \omega_{k\lambda}^2 \left( \hat{a}_{k\lambda} \hat{a}^{\dagger}_{k\lambda} + \hat{a}^{\dagger}_{k\lambda} \hat{a}_{k\lambda} \right) = \sum_{k,\lambda} \hbar \omega_{k\lambda} \left( \hat{a}_{k\lambda} \hat{a}^{\dagger}_{k\lambda} + \hat{a}^{\dagger}_{k\lambda} \hat{a}_{k\lambda} \right).
\end{equation}
De igual forma, los campos se cuantizan obteniendo \cite{Agarwal_2012}
\begin{equation}
  \hat{\mathbf{A}} =  \sum_{k,\lambda} \left( \frac{2\hbar \pi c^2}{V\omega_k} \right)^{1/2} \left[ \hat{a}_{k\lambda} \mathbf{u}_{k\lambda} (\mathbf{r})e^{-i\omega_k t} + \hat{a}^{\dagger}_{k\lambda} \mathbf{u}_{k\lambda}^* (\mathbf{r})e^{i\omega_{k} t} \right],
\end{equation}
\begin{align}
  \hat{\mathbf{E}} & = \sum_{k,\lambda} \left( \frac{2\hbar \pi c^2}{V\omega_k} \right)^{1/2} \frac{i\omega_{k}}{c} \left[ \hat{a}_{k\lambda} \mathbf{u}_{k\lambda} (\mathbf{r})e^{-i\omega_k t} -  \hat{a}^{\dagger}_{k\lambda} \mathbf{u}_{k\lambda}^* (\mathbf{r})e^{i\omega_{k} t} \right] \nonumber \\
                   & = i \sum_{k,\lambda} \left( \frac{2\hbar \pi \omega_k}{V} \right)^{1/2} \left[ \hat{a}_{k\lambda} \mathbf{u}_{k\lambda} (\mathbf{r})e^{-i\omega_k t} -  \hat{a}^{\dagger}_{k\lambda} \mathbf{u}_{k\lambda}^* (\mathbf{r})e^{i\omega_{k} t} \right],
\end{align}
\begin{equation}
  \hat{\mathbf{B}} = i \sum_{k,\lambda} \left( \frac{2\hbar \pi \omega_k}{V} \right)^{1/2} \left[ \mathbf{\hat{k}} \times \mathbf{u}_{k\lambda}(\mathbf{r}) \hat{a}_{k\lambda} e^{-i\omega_k t} - \mathbf{\hat{k}} \times \mathbf{u}_{k\lambda}^* (\mathbf{r}) \hat{a}^{\dagger}_{k\lambda} e^{i\omega_{k} t} \right].
\end{equation}