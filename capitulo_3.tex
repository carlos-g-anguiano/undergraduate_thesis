\chapter{Oscilador Arm\'onico Cu\'antico}
\section{Oscilador armónico cuántico}

El oscilador armónico cuántico (OAC) es uno de los pocos modelos que tiene soluciones analíticas en la mecánica cuántica. La cinemática de varios sistemas periódicos y de tipo ondulatorio se puede describir utilizando el modelo del OAC.
Varios potenciales independientes del tiempo $V(x)$ se pueden describir alrededor de un mínimo con una expansión en serie de Taylor
\begin{align*}
V(x) & = \sum_{n=0}^{\infty} \frac{V^{(n)}}{n!} (x)|_{x_0} (x-x_0)^n \\ &= V(x_0) + \frac{dV}{dx}\Big|_{x_0} (x-x_0) + \frac{1}{2}\frac{d^2V}{dx^2}\Big|_{x_0} (x-x_0)^2 + \dots \,,
\end{align*}
el segundo término se anula por definición de mínimo local. Por formalismo, se puede recorrer el potencial $x = x' + x_0$ para que el potencial en $x_0$ sea $V(x_0) = 0$, eliminando el primer término. Finalmente, el término dominante en la expansión es la segunda derivada
\begin{equation}
\label{OA.1}
V(x) = \frac{1}{2}\frac{d^2V(x)}{dx^2}\Big|_{x_0}x^2 \,,
\end{equation}

lo anterior se puede aplicar para un potencial que depende de más dimensiones $V(\mathbf{x})$. Asumiendo que todos los potenciales se han recorrido de manera análoga $x_i \to x_i + x_{oi}$

\begin{equation*}
  V(x_1, \dots, x_N) = \frac{1}{2}\sum_{i=1}^N \sum_{j=1}^N \frac{\partial^2 V}{\partial x_i \partial x_j}\Big|_{0} x_i x_j \,,
\end{equation*}

para hamiltonianos con potenciales cuadráticos, es siempre posible hacer un cambio de coordenadas de la forma

\begin{equation*}
  (x_{1}, \dots, x_N) \to (u_1, \dots, u_N)\,,
\end{equation*}

donde el potencial se desacopla, \textit{i.e.} sin derivadas cruzadas, y se puede describir por $n$ osciladores armónicos individuales

\begin{equation*}
  V(u_1, \dots, u_N) = \frac{1}{2} \sum_{i=1}^{N} \frac{\partial^2 V}{\partial u_i^{2}}\Big|_0 u_i^2\,.
\end{equation*}

Considerando un ejemplo básico de un oscilador armónico clásico, se tienen las ecuaciones clásicas de movimiento

\begin{align}
  m \frac{d^2x}{xt^2} & = -kx \,, \label{OA.2}           \\
  p                   & = m\frac{dx}{dt} \,, \label{OA.3}
\end{align}

con $k$ la constante del resorte, $m$ la masa de la partícula, $x$ el desplazamiento de la posición de equilibrio y $p$ el momento lineal, esto se sustituye en la ecuación estacionaria de Schrödinger, que se debe resolver es:

\begin{equation}
\label{OA.4}  
-\frac{\hbar^2}{2m}\frac{d^2}{dx^2}\psi(x) + \frac{1}{2}m \omega^2 x^2 \psi(x) = E \psi(x) \,.
\end{equation}

Una partícula en un sistema tiene una energía cinética que corresponde a
\begin{equation*}
\hat{T} = \frac{\hat{p}^2}{2m}\,,
\end{equation*}

y el hamiltoniano correspondiente está dado por
\begin{equation}
  \hat{H} = \hat{T} + \hat{V} = \frac{\hat{p}^2}{2m} +\frac{1}{2} m\omega^2 \hat{q}^2 \label{OA.5}\,,
\end{equation}

% Following the Loudon method

donde $\omega$ es la frecuencia angular del oscilador clásico.
El problema del oscilador armónico puede ser resuelto de forma analítica (encontrando los eigenvalores y eigenfunciones de la ecuación de Schrödinger) o de forma algebraica, que involucra la introducción de dos operadores $\hat{a}^{\dagger}$ y $\hat{a}$, los operadores escalera de creación y aniquilación, respectivamente, definidos en términos de $\hat{q}$ y $\hat{p}$ como:
% Definido en Sakurai de esta forma

\begin{align}
\hat{a}^{\dagger} & = (2m\hbar\omega)^{-1/2}(m\omega \hat{q} - i\hat{p})\label{OA.6} \,,  \\
\hat{a} & = (2m\hbar \omega)^{-1/2}(m\omega \hat{q} + i \hat{p})\label{OA.7} \,,
\end{align}
el conmutador de estos operadores resulta
\begin{equation}
\label{OA.8}	
[\hat{a}, \hat{a}^{\dagger}] = \hat{1} \,,
\end{equation}
los operadores no son hermitianos, y como tal no representan una propiedad observable del oscilador. % Loudon

Los productos entre estos dos operadores son
\begin{align}
\hat{a}^{\dagger}\hat{a} & = \frac{1}{2m\hbar\omega} \left( \hat{p}^2 + m^2\omega^2\hat{q}^2 + im\omega\hat{q}\hat{p} - im\omega\hat{p}\hat{q} \right) \nonumber     \\
& = \frac{1}{\hbar\omega}\left( \frac{\hat{p}^2}{2m} + \frac{1}{2}m\omega^2\hat{q}^2 +\frac{i\omega}{2} [\hat{q}, \hat{p}] \right) 
\nonumber\\
& = \frac{1}{\hbar\omega} \left( \hat{H} + \frac{i\omega}{2}(i\hbar) \right) \nonumber \\ 
&= \frac{1}{\hbar\omega} \left( \hat{H} - \frac{1}{2}\hbar\omega \right) \label{OA.9}\,.
\end{align}
A partir del anticonmutador de los operadores (\ref{OA.5}) y (\ref{OA.6}), se obtiene lo siguiente
\begin{align}
\frac{1}{2}\{ \hat{a},\hat{a}^{\dagger}\} & = \frac{1}{2}\hat{a}\hat{a}^{\dagger} + \frac{1}{2}\hat{a}^{\dagger}\hat{a}                                                \nonumber \\
& = \frac{1}{2}\hat{a}\hat{a}^{\dagger} - \frac{1}{2}\hat{a}^{\dagger}\hat{a} +\frac{1}{2}\hat{a}^{\dagger}\hat{a} + \frac{1}{2}\hat{a}^{\dagger}\hat{a} \nonumber\\
& = \frac{1}{2}[\hat{a},\hat{a}^{\dagger}] + \hat{a}^{\dagger}\hat{a}                                                        \nonumber \\
& = \frac{1}{2} + \hat{a}^{\dagger}\hat{a} \label{OA.10}\,,
\end{align}

por lo que el de las ecuaciones (\ref{OA.9}) y (\ref{OA.10}) el hamiltoniano (\ref{OA.5}) se puede reescribir como

\begin{equation}
\label{OA.11}
\hat{H} = \hbar \omega \left(\hat{a} \hat{a}^{\dagger} + \frac{1}{2}\right) = \frac{1}{2}\hbar \omega \left( \hat{a}\hat{a}^{\dagger} + \hat{a}^{\dagger}\hat{a} \right) \,.
\end{equation}

Se introduce también el operador número, definido como
\begin{equation}
\label{OA.12}
\hat{n} = \hat{a}^{\dagger} \hat{a}\,,
\end{equation}
y el hamiltoniano (\ref{OA.11}) se puede escribir también como

\begin{equation}
\label{OA.13}
\hat{H} = \hbar \omega \left(\hat{n} + \frac{1}{2}\right)\,,
\end{equation}
cuya ecuación de eigenvalores de la energía es
\begin{equation}
\label{OA.14}
\hat{H} \ket{n} = E_n\ket{n} \,,
\end{equation}
donde $\ket{n}$ es un eigenestado del OAC con eigenvalor $E_n$. A partir de ciertas operaciones algebráicas, se puede encontrar otra ecuación de eigenvalores a partir de la anterior multiplicando $\hat{a}$ por la izquierda, y resulta de la forma
\begin{align*}
\hat{H} \hat{a}\ket{n} &= \hbar\omega\left(\hat{a}^{\dagger}\hat{a}+\frac{1}{2}\right)\hat{a}\ket{n} = \hbar\omega\left(\hat{a}\hat{a}^{\dagger}+\hat{a}^{\dagger}\hat{a}-\hat{a}\hat{a}^{\dagger}+\frac{1}{2}\right)\hat{a}\ket{n} \\
& = \hbar\omega\left(\hat{a}\hat{a}^{\dagger}-\left[\hat{a},\hat{a}^{\dagger}\right]+\frac{1}{2}\right)\hat{a}\ket{n} = \hbar\omega\left(\hat{a}\hat{a}^{\dagger}-1+\frac{1}{2}\right)\hat{a}\ket{n}\\
& = \hbar\omega\left(\hat{a}\hat{a}^{\dagger}\hat{a}-\hat{a}+\frac{1}{2}\hat{a}\right)\ket{n} = \hat{a}\hbar\omega\left(\hat{a}^{\dagger}\hat{a}+\frac{1}{2}-1\right)\ket{n}\\
& = \hat{a}\left(\hbar\omega\left(\hat{n}+\frac{1}{2}\right)-\hbar\omega\right)\ket{n} = \hat{a}\left(\hat{H}-\hbar\omega\right)\ket{n} = \left(E_{n}-\hbar\omega\right)\hat{a}\ket{n}\,,
\end{align*}
donde $\hat{a}\ket{n}$ es también un eigenestado del sistema, pero ahora con un valor propio de energía $E_n - \hbar\omega$, los cuales se denotarán como $\ket{n-1}$ y $E_{n-1}$ respectivamente. Así, la nueva ecuación de eigenvalores es \cite{Loudon}
\begin{equation}
\label{OA.15}
\hat{H}\, \hat{a}\ket{n} = E_{n-1}\, \hat{a}\ket{n}\,,
\end{equation} % Citar a Loudon para este procedimiento
de manera similar, pero ahora aplicando el operador $\hat{a}^{\dagger}$ por la izquierda, se llega a la ecuación
\begin{equation}
\label{OA.16}
\hat{H} \,\hat{a}^{\dagger}\ket{n} = E_{n+1}\, \hat{a}^{\dagger}\ket{n}\,.
\end{equation}
Los operadores escalera nos permiten conocer el resto de los valores de la energía del oscilador armónico una vez conocido $E_n$, estos valores varían en incrementos de $\hbar\omega$. Debido a que no puede haber un estado con energía negativa, se propone un estado $\ket{0}$, llamado estado base, al que corresponda una energía mínima $E_0$, y con la propiedad:
\begin{equation}
\label{OA.17}
\hat{a}\ket{0} = 0\,.
\end{equation}
A partir de esta propiedad se determina que $E_0 = \frac{1}{2}\hbar\omega$ y en general
\begin{equation}
\label{OA.18}
E_n = \left( n + \frac{1}{2} \right) \hbar\omega, \quad n\in\mathbf{N}\,.
\end{equation}
% ============================
% ============================
% ============================
% Add explanation of how the cavity modes of EM field expansion = harmonic oscillator = why problems reduce to the harmonic oscillator (what properties let us do that simplification)
% ============================
% ============================
% ============================
Los estados número son eigenestados simultáneos del operador hamiltoniano (\ref{OA.13}) y del operador número (\ref{OA.12}), por lo que tienen una base de eigenvectores en común, donde
\begin{equation}
\label{OA.19}
\hat{n}\ket{n} = n\ket{n}\,.
\end{equation}

Los estados se pueden normalizar utilizando las condiciones
\begin{equation}
\label{OA.20}	
\langle n\vert m\rangle = \delta_{n,m}
\end{equation}
y los coeficientes de normalización para la aplicación de los operadores escalón resultan
\begin{align}
\hat{a}\ket{n} &= \sqrt{n} \ket{n-1} \,,\label{OA.21}\\
\hat{a}^{\dagger}\ket{n} &= \sqrt{n+1} \ket{n+1} \,.\label{OA.22}
\end{align}

Los operadores $\hat{a}^{\dagger}$ y $\hat{a}$ no son hermitianos, por lo que no representan alguna magnitud f\'isica observable. Es conveniente entonces definir, a partir de la definición de operadores posición $\hat{q}$ y momento $\hat{p}$ generalizado en su forma adimensional utilizados en la expresión del Hamiltoniano del oscilador armónico, los operadores de cuadratura

\begin{equation}
\label{OA.23}
\hat{X} = \sqrt{\frac{m\omega}{2\hbar}} \, \hat{q} = \frac{1}{2}\left( \hat{a} + \hat{a}^{\dagger} \right)\,,
\end{equation}
\begin{equation}
\label{OA.24}
\hat{Y} = \frac{1}{\sqrt{2m\hbar\omega}} \, \hat{p} = \frac{i}{2}\left( \hat{a}^{\dagger} - \hat{a} \right)\,,
\end{equation}
los operadores escalera en términos de los de cuadratura son
\begin{equation}
\label{OA.25}	
\hat{a} = \hat{X} + i{Y}\,,
\end{equation}
\begin{equation}
\label{OA.26}
\hat{a}^{\dagger} = \hat{X} - i\hat{Y}\,,
\end{equation}
la relación de conmutación entre estos operadores es
\begin{equation}
\label{OA.27}
\left[ \hat{X}, \hat{Y} \right] = \frac{i}{2}
\end{equation}
y su relación de incertidumbre es:
\begin{equation}
\label{OA.28}
\langle (\Delta X)^2 \rangle \langle (\Delta Y)^2 \rangle \geq \frac{1}{16}\,,
\end{equation}
con estos también se puede expresar el Hamiltoniano del oscilador armónico simple
\begin{equation}
\label{OA.29}
\hat{H} = \hbar \omega \left( \hat{X}^2 + \hat{Y}^2 \right)\,.
\end{equation}
% Agregar teoría de fasores e incertidumbre de Fox cap 7.3