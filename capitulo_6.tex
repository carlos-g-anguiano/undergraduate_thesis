\chapter{Grupos y \'Algebra de Lie, Grupo de Heisenberg-Weyl}
\section{Grupos y Álgebras de Lie, Grupo de Heisenberg-Weyl}

El teorema de Noether estipula que correspondiente a cada ley de conservación existe una simetría diferenciable. En particular establece una conexión entre el generador de una transformación de simetría con la cantidad conservada. El estudio de estas cantidades conservadas es la base de la física, pues describen a los sistemas en sí. En la mecánica cuántica se puede ver como generador de traslaciones en el espacio $-i\delta_i$ al momento $\hat{p}_i$, mientras que la energía $\hat{E}$ como cantidad conservada proviene de la invarianza bajo la acción del generador de traslaciones temporales $i\delta_0$. La posición no tiene relacionada una simetría o ley de conservación, por lo que simplemente se utiliza el operador de posición $\hat{x}_i$. (Schwichtenberg)

Las cantidades físicas de la teoría son representadas por operadores diferenciales. Para entender a detalle cómo funcionan los operadores, sobre qué actúan y cómo se relacionan a cantidades físicas medibles primero se tiene que estudiar las simetrías, que son la invarianza bajo alguna transformación y que es el objeto de estudio de la teoría de grupos. El grupo de Poincaré es el conjunto de todas las transformaciones permitidas por estas restricciones, donde el objeto invariante es la métrica de Minkowski en el espacio-tiempo.

Un grupo es una colección de transformaciones. Formalmente es un conjunto $G$ con una operación binaria $\circ$ de finida en $G$ que satisface cuatro condiciones
\begin{enumerate}
  \item Sean $g_1,g_2 \in G$ dos elementos del grupo, entonces el producto de $g_1$ y $g_2$ bajo $\circ$ pertenece también al grupo, es decir $g_1 \circ g_2 \in G$.
  \item Existe un elemento $e\in G$ tal que para toda $g\in G$, $g\circ g = g$. A este elemento se le llama identidad.
  \item Para cada $g\in G$ existe un elemento $g^{-1}$ llamado inverso que cumple $g\circ g^{-1} = g^{-1}\circ g =e$
  \item Asociatividad. Para todo $g_1, g_2, g_3 \in G$, $g_1 \circ (g_2 \circ g_3) = (g_1 \circ g_2) \circ g_3$
\end{enumerate}

Las transformaciones bajo las cuales se conserva una simetría pueden ser continuas o discretas. Considere el caso de una rotación de un cuadrado con aristas de longitud $2a$ y centrado en $x,y=0$ en el plano cartesiano. Matemáticamente un cuadrado es un conjunto de puntos, y su simetría es una transformación que  Al rotar el cuadrado por una cantidad menor de $pi/2$ radianes en dirección contrarreloj respecto al origen sobre el plano $xy$, las nuevas posiciones de los vértices no coinciden con las posiciones de estos previo a la rotación. Sin embargo, cuando se hace una rotación de $n\pi/2$ radianes con $n\in \mathbb{Z}$, los vértices se alinearán con las posiciones iniciales, y bajo esta transformación es virtualmente imposible diferenciar entre el cuadrado previo o posterior a esta. Se dice entonces que el sistema es simétrico ante rotaciones discretas de $n\pi/2$ radianes.

Por otro lado, considere ahora una circunferencia de radio $a$ centrada en $x,y=0$. Rotando sobre el plano $xy$ respecto al origen cualquier cantidad de $\phi$ radianes será imposible distinguir entre la circunferencia rotada y la no transformada. El anterior es un ejemplo de una simetría continua, que es el objeto de estudio del álgebra de Lie.

De manera más formal, sea $\mathbf{a}$ un vector y $O$ una transformación, que mantenga la norma de este, es decir que $|\mathbf{a}| = |O\mathbf{a}|$. La transformación tiene que ser ortogonal, es decir $O^{T}O = I$. Las transformaciones ortogonales en dos dimensiones se denotan por $O(2)$. Estas transformación también cumplen $\det(O) = \pm 1$, esta característica les da el nombre de grupo especial ortogonal, y se denota por $SO(2)$. Una matriz de rotación en dos dimensiones por un ángulo $\theta$ se representa matricialmente como
\begin{equation*}
  R_\theta = \begin{pmatrix}
    \cos(\theta) & -\sen(\theta) \\
    \sen(\theta) & cos(\theta)
  \end{pmatrix}
\end{equation*}

Por otro lado, los números complejos junto a su operación de multiplicación forman un grupo unitario llamado $U(1)$, el cuál se caracteriza por la condición $U^\dagger U = 1$. La transformación se escribe usando el parámetro $\theta$ como
\begin{equation*}
  R_\theta = e^{i\theta} = \cos(\theta) + i\sen(\theta)
\end{equation*}
que cumple la característica de ser unitaria.

Las anteriores dos transformaciones describen rotaciones, y son un ejemplo de dos representaciones que actuan de la misma manera, formando un isomorfismo entre $SO(2)$ y $U(1)$. La representación matricial de los números complejos es entonces
\begin{equation*}
  1 = \begin{pmatrix}
    1 & 0 \\ 0 & 1
  \end{pmatrix} \quad i = \begin{pmatrix}
    0 & -1 \\ 1 & 0
  \end{pmatrix}
\end{equation*}
Para el caso de la circunferencia, se puede realizar una rotación con un argumento tan pequeño como se guste, una rotación arbitrariamente cercana a la identidad, a diferencia de los grupos discretos  donde el parámetro de transformación no es continuo y por tanto no hay elementos arbitrariamente cercanos a la identidad. Esta transformación infinitesimal se define como
\begin{equation*}
  g(\epsilon) = I + gX
\end{equation*}
donde $\epsilon>0$ es un número pequeño y $X$ es un objeto llamado generador. Una transformación infinitesimal aplicada de manera sucesiva genera una transformación finita $h(\theta) = \prod_k (I+\epsilon X)$, con $\theta$ un parámetro de transformación finita. Se puede reescribir $\epsilon$ en términos del parámetro de transformación finita $\theta$ dividiendo este último entre un número entero arbitrariamente grande $N$. El generador es entonces
\begin{equation*}
  g(\theta) = I = \frac{\theta}{N} X
\end{equation*}
Aplicando esta transformación $N$ veces, y aplicando el límite $N\to\infty$ se genera una transformación finita
\begin{equation*}
  h(\theta) = \lim_{N\to\infty}\left( I + \frac{\theta}{N}X \right)^N = e^{\theta X}
\end{equation*}
observamos entonces que la transformación de argumento finito $\theta$ es generada por $X$. Para calcular el generador se obtiene la derivada de la transformación finita respecto al argumento y se evalúa en $\theta=0$
\begin{equation}\label{eq:generator}
  X = \frac{dh(\theta)}{d\theta} \bigg|_{\theta=0}
\end{equation}
los generadores brindan información importante sobre los grupos que se estudian.
En el caso particular de un grupo de transformaciones dado por matrices, se puede calcular la expansión en series de Taylor de un elemento del grupo cercano a la identidad
\begin{equation*}
  h(\theta) = I + \frac{dh}{d\theta}\bigg|_{\theta=0}\theta + \frac{1}{2}\frac{d^2h}{d\theta^2}\bigg|_{\theta=0}\theta^2 + \dots = \sum_{n=0}^{\infty} \frac{1}{n!}\frac{d^n h}{h\theta^n}\bigg|_{\theta=0}\theta^n
\end{equation*}
Para grupos de matrices de Lie se define un correspondiente álgebra de Lie como una colección de objetos que generan un elemento del grupo cuando se calculan sus exponenciales. Para un grupo de Lie $G$ dado por matrices de dimensión $n\times n$, el álgebra $\mathfrak{g}$ de $G$ está dada por aquellas matrices $n\times n$ $X$ tales que $e^{tX}\in G$ para $t\in \mathbb{R}$, junto a una operación llamada corchete de Lie $[,]$ que define cómo se combinan estas matrices. La multiplicación de dos elementos del álgebra de Lie no necesariamente es un elemento de este mismo, es decir, no es cerrada bajo la multiplicación del grupo, pero si lo es bajo el corchete de Lie. La relación entre el álgebra de Lie y el grupo de Lie está dada por la fórmula de Baker-Campbell-Hausdorff
\begin{equation*}
  e^X \circ e^Y = e^{X+Y+\frac{1}{2}[X,Y] + \frac{1}{12}[X,[X,Y]] - \frac{1}{12}[Y,[X,Y]] + \dots}
\end{equation*}
del lado izquierdo, hay dos elementos del grupo escritos en término de los generadores $X,Y\in \mathfrak{g}$, mientras que del lado derecho se tiene un solo elemento del grupo y la multiplicación de elementos del grupo se ha transformado en una suma de elementos del álgebra de Lie. En el caso particular de matrices, el corchete de Lie es el conmutador $[X,Y] = XY-YX$.
De manera formal, un álgebra de Lie es un espacio vectorial $\mathfrak{g}$ junto a una operación binaria $[,]:\mathfrak{g}\times \mathfrak{g} \to \mathfrak{g}$. La operación binaria satisface los siguientes axiomas:
\begin{enumerate}
  \item Bilinearidad: $[aX + bY, Z] = a[X,Y] + b[Y, Z]$ y $[Z, aX + bY] = a[Z,X] + b[Z,Y]$ para $a,b\in \mathbb{F}$ con $\mathbb{F}$ un cuerpo, y se cumple para todo $X,Y,Z\in \mathfrak{g}$
  \item Anti-conmutatividad: $[X,Y] = -[Y,X]$, $\forall X, Y \in \mathfrak{g}$
  \item Satisfacen la identidad de Jacobi: $[X, [Y,Z]] + [Z,[X,Y]] + [Y,[Z,X]] = 0$, $\forall X,Y,Z\in \mathfrak{g}$
\end{enumerate}
Diferentes grupos pueden tener el mismo álgebra de Lie, la diferencia d estos es cómo los generadores de estos grupos se comportan bajo el álgebra de Lie.
Considerando el ejemplo particular de las matrices de rotación $SO(3)$, recordando que estas están dadas por
\begin{equation*}
  R_x = \begin{pmatrix}
    1 & 0            & 0             \\
    0 & \cos(\theta) & -\sen(\theta) \\
    0 & \sen(\theta) & \cos(\theta)
  \end{pmatrix} \quad
  R_y = \begin{pmatrix}
    \cos(\theta)  & -\sen(\theta) & 0            \\
    0             & 1             & 0            \\
    -\sen(\theta) & 0             & \cos(\theta)
  \end{pmatrix}
\end{equation*}
\begin{equation*}
  R_z = \begin{pmatrix}
    \cos(\theta) & -\sen(\theta) & 0 \\
    \sen(\theta) & \cos(\theta)  & 0 \\
    0            & 0             & 1
  \end{pmatrix}
\end{equation*}
Las condiciones de estas transformaciones son las mismas que las rotaciones en dos dimensiones, es decir, son ortogonales $O^T O = I$ y su determinante es la unidad $\det(O) = 1$. Un elemento cualesquiera del grupo en términos de un generador $J$ está dado por $O = e^{\theta J}$. Sustituyendo esto en las condiciones de las transformaciones se obtiene que $J^T + J = 0$ y $\mathrm{Tr}(J) = 0$. A partir de estas condiciones se pueden escribir tres elementos linealmente independientes que formen una base para $SO(3)$, considere
\begin{equation*}
  J_1 = \begin{pmatrix}
    0 & 0 & 0 \\ 0 & 0 & -1 \\ 0 & 1 & 0
  \end{pmatrix} \quad
  J_2 = \begin{pmatrix}
    0 & 0 & 1 \\ 0 & 0 & 0 \\ -1 & 0 & 0
  \end{pmatrix} \quad
  J_3 = \begin{pmatrix}
    0 & -1 & 0 \\ 1 & 0 & 0 \\ 0 & 0 & 0
  \end{pmatrix} \quad
\end{equation*}
Así, cualquier generador se puede escribir de la forma $J = aJ_1+ bJ_2+cJ_3$. Las entradas de las bases se pueden representar por
\begin{equation*}
  (J_i)_{jk} = -\epsilon_{ijk}
\end{equation*}
Como en este caso se conocen de antemano las matrices de rotación, se puede ocupar la fórmula (\ref{eq:generator}) para deducir $J_i$ de cada matriz de rotación $R_i$, para $J_1$ por ejemplo
\begin{equation*}
  J_1 = \frac{dR_1}{d\theta}\bigg|_{\theta=0} = \begin{pmatrix}
    0 & 0 & 0 \\ 0 & -\sen(\theta) & -cos(\theta)\\ 0 & \cos(\theta) & -\sen(\theta)
  \end{pmatrix} \Bigg|_{\theta=0} = \begin{pmatrix}
    0 & 0 & 0 \\ 0&0&-1\\ 0&1&0
  \end{pmatrix}
\end{equation*}
(Physics from Symmetry - Schwichtenberg)
Los operadores que se utilizan con frecuencia para describir sistemas cuánticos con un grado de libertad son la posición $\hat{q}$ y momento $\hat{p}$ y tienen las relaciones de conmutación de Heisenberg $[\hat{q}, \hat{p}] = i\hbar \hat{I}$, $[\hat{q}, \hat{I}] = [\hat{p}, \hat{I}]=0$ con $\hat{I}$ el operador identidad. Estos operadores actuan sobre un espacio de Hilbert $\mathcal{H}$. A partir de estos operadores se definieron los operadores escalera $\hat{a}$ y $\hat{a}^{\dagger}$, que siguen las relaciones de conmutación $[\hat{a}, \hat{a}^{\dagger}] = \hat{I}$, $[\hat{a}, \hat{I}]=[\hat{a}^{\dagger}, \hat{I}] = 0$. Estas relaciones de conmutación entre los operadores $\hat{q}$, $\hat{p}$ y la identidad $\hat{I}$ (y respectivamente $\hat{a}$, $\hat{a}^{\dagger}$ y $\hat{I}$) indica que estos son generadores de un álgebra de Lie $\mathcal{W}_1$, a la que se le llama álgebra de Heisenberg-Weyl. Este álgebra tridimensional se define usando las siguientes cantidades
\begin{equation*}
  e_1 = i\frac{\hat{p}}{\sqrt{\hbar}}, \quad e_2 = i\frac{\hat{q}}{\sqrt{\hbar}}, \quad e_3 = i\hat{I}
\end{equation*}
que, además de ser elementos abstractos del álgebra de Lie, son a su vez operadores en un espacio de Hilbert. Las relaciones de conmutación de este álgebra son
\begin{equation*}
  [e_1, e_2] = e_3, \quad [e_1, e_3] = [e_2, e_3] = 0
\end{equation*}
Un elemento $x$ del álgebra de Lie se representa por una combinación lineal $x = x_1e_1 + x_2e_2 + s e_3$ o una 3-tupla de números reales $(s;x_1,x_2)$. Se eligen $x_1 = -Q/\sqrt{h}$ y $x_2 = P/\sqrt{h}$, y en términos de los generadores del álgebra de Lie se escribe al elemento $x$ como
\begin{equation*}
  x = \frac{i}{\hbar}(P\hat{q} - Q\hat{p}) is\hat{I}
\end{equation*}
o, en términos de los operadores escalera, considerando que $\hat{q} = \sqrt{\hbar/2}(\hat{a} + \hat{a}^{\dagger})$ y $\hat{p} = -i\sqrt{\hbar/2}(\hat{a} - \hat{a}^{\dagger})$
se tiene
\begin{equation*}
  x = \left( \frac{Q+iP}{\sqrt{2\hbar}} \right)\hat{a}^{\dagger} - \left( \frac{Q-iP}{\sqrt{2\hbar}} \right)\hat{a} + is\hat{I}
\end{equation*}
definiendo $\alpha =  \frac{Q+iP}{\sqrt{2\hbar}}$ la expresión se reduce a
\begin{equation*}
  x = \alpha\hat{a} - \alpha^* \hat{a}^{\dagger} + is\hat{I}
\end{equation*}
El conmutador de dos elementos $x = (s_x; x_1, x_2)$ y $y = (s_y, y_1, y_2)$ del álgebra está dado por $[x,y] = (x_1 y_2 - x_2 y_1)e_3$. Finalmente, se puede construir el grupo de Lie a partir del álgebra por exponenciación
\begin{equation*}
  e^{x} = e^{is\hat{I}}D(\alpha), \quad D(\alpha) = e^{\alpha\hat{a}^{\dagger} - \alpha^* \hat{a}}
\end{equation*}
$D(\alpha)$ se define como el operador desplazamiento.

% Search in Perelomov for the connection between lie algebra and the single mode operator