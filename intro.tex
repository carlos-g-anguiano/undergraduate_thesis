



\chapter{Introducci\'on}


La luz exhibe tanto aspectos de onda como de partícula. Se propaga en el espacio e interfiere consigo misma, se dispersa en medios ópticos como los lentes y muestra efectos de polarización. Todas estas propiedades son generalmente consideradas como características clásicas de las ondas, ya que se deducen de las ecuaciones de Maxwell \cite{purcell2011electricity,Jackson:100964}. También podríamos afirmar que la luz se comporta como partículas en movimiento, pero que, a pesar de esto, sigue las reglas de la interferencia de las ondas. Esta representación inusual ha desconcertado a innumerables personas a lo largo de gran parte de este siglo. Hablando estrictamente, esta imagen aún no ha sido completamente explicada; más bien, ha sido formulada de manera más precisa en la teoría cuántica de la luz. Esto se debe a que, cuando se detecta con suficiente precisión, la luz aparece como pulsos discretos de detección denominados fotones \cite{Leonhardt,Bachor}, lo cual no necesariamente implica que los aspectos de partícula sean completamente cuánticos.\\


Para aumentar aún más la ambigüedad de todas estas palabras, surge la siguiente pregunta: ?`La propagación de las ondas es un proceso clásico o cuántico? En otras palabras, ?`La propagaci\'on se basa exclusivamente en los aspectos más fundamentales de las características clásicas de las ondas y las ecuaciones de Maxwell, o en la maquinaria sofisticada de los aspectos cuánticos y el modelo del oscilador armónico?\\









Para hacer todas estas palabras ambiguas más precisas y para resumir la historia, postulamos que la intensidad del campo eléctrico $\hat{E}$ del campo de luz está dada por..



 Este libro se enfoca en los aspectos cuánticos de la luz. Utilizaremos el concepto más primitivo para las características de onda clásicas, pero una maquinaria sofisticada para los aspectos cuánticos. Nuestro modelo es el oscilador electromagnético. Una función de vector complejo $u(x,t)$ llamada modo espaciotemporal abarca todos los aspectos de onda clásicos, incluida la polarización. El ejemplo más simple de un modo espaciotemporal es una onda plana.















